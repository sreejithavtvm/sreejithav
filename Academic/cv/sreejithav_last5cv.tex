 % LaTeX resume using res.cls
\documentclass[margin]{res}
\usepackage{amsfonts}

%\usepackage{helvetica} % uses helvetica postscript font (download helvetica.sty)
%\usepackage{newcent}   % uses new century schoolbook postscript font 
\setlength{\textwidth}{5.1in} % set width of text portion
\usepackage{etaremune}
\usepackage{hyperref}
\hypersetup{
hidelinks
}

\begin{document}

% Center the name over the entire width of resume:
 \moveleft.5\hoffset\centerline{\Large \bf A. V. Sreejith} 
% Draw a horizontal line the whole width of resume:
 \moveleft\hoffset\vbox{\hrule width\resumewidth height 1pt}\smallskip
% address begins here
% Again, the address lines must be centered over entire width of resume:
\moveleft.5\hoffset\centerline{Indian Institute of Technology Goa}
\moveleft.5\hoffset\centerline{F6, Academic block, GEC campus, Farmagudi, Goa, India - 403401}
\moveleft.5\hoffset\centerline{E-mail: sreejithav@iitgoa.ac.in}
\moveleft.5\hoffset\centerline{ \href{http://www.iitgoa.ac.in/~sreejithav/} {\tt{www.iitgoa.ac.in/$\sim$sreejithav/}}}
%\moveleft.5\hoffset\centerline{Phone: +918939601670}

 
\begin{resume}
 
\section{Broad Subject}  Computer Science, Logic and formal verification
%\section{Personal Information}
%\begin{tabular}{rl}
%Date of Birth: & March 21, 1983 \\
%Gender: & Male \\
%Citizenship: & Indian \\ 
%Permanent Address: & TC 14/1328, Kochu Madom, Vazhuthacaud, \\
% & Thiruvananthapuram, Kerala-695014
%\end{tabular}

\section{Research Experience} 
					Assistant Professor at \\
					Indian Institute of Technology (IIT) Goa \hfill October 2017 to at present \\
					\noindent\rule{13cm}{0.4pt} \\
					Postdoctoral Researcher at \hfill  \hfill September 2016 to January 2017 \\
					Institute of Informatics, University of Warsaw \hfill July 2017 to September 2017 \\
%					Mentor: Prof. Miko\l{}aj Boja\'nczyk \\
					\noindent\rule{13cm}{0.4pt} \\
					Visiting Faculty at 	\hfill December 2015 to August 2016 \\
					Chennai Mathematical Institute (CMI)  \hfill  February 2017 to June 2017 \\
					\noindent\rule{13cm}{0.4pt} \\
					Postdoctoral Researcher  at Laboratoire d{'}Informatique Algorithmique: \hfill April 2014\\
					Fondements et Applications (LIAFA),	Paris, France \hfill to September 2015 \\
%					Mentor: Prof. Thomas Colcombet \\
					\noindent\rule{13cm}{0.4pt} \\
					Postdoctoral Researcher  at Tata Institute of  Fundamental \hfill September 2013 \\
					Research (TIFR), Mumbai	\hfill	to March 2014
					\noindent\rule{13cm}{0.4pt} \\					
					Research Position with Dr. Klaus-J\"orn Lange, Wilhelm-Schickard \hfill March 2012\\
					-Institut f\"ur Informatik, University of T\"ubingen, Germany \hfill to August 2012 
					\noindent\rule{13cm}{0.4pt} \\
					PhD in Mathematics at \hfill August 2007\\
					Institute of Mathematical Sciences (IMSc), Chennai \hfill to September 2013\\
					Advisor: Prof. Kamal Lodaya  \\
					Thesis: Regular Quantifiers in Logics  \\
					\noindent\rule{13cm}{0.4pt} 


\section{Awards}
\href{http://awards.acm.org/award_winners/sreejith_9923861.cfm}{\tt{\emph{\bf ACM India Doctoral Dissertation Award 2014: Honourable mention}}} for doctoral dissertation, ``Regular Quantifiers in Logics". 
%The award is for the second best doctoral dissertation in computer science in India for the year 2014. It is given by the Association for computing machinery (ACM), India.
%{$\bullet$} First prize in programming contest conducted by Cochin University in 2002. \\
%{$\bullet$} Regional Mathematical Olympiad (RMO) in 10th standard.

\section{Industry Experience}
Software Engineer, Infosys Technologies, India \hfill August 2004 to July 2005
	
%\newpage
\section{Education} 
					MTech in Computer Science \hfill August 2005 to May 2007\\
					Institute: Indian Institute of Technology (IIT), Madras \hfill CGPA: 8.86 \\
%					Project: Automatic Raga classification \\
%					Project guide: Prof. C. Chandra Sekhar \hfill CGPA: 8.86 \\
					\noindent\rule{13cm}{0.4pt} \\
					BTech in Computer Science \hfill August 2000 to June 2004 \\
					Institute: College of Engineering, Thiruvananthapuram \hfill Marks: 73\% \\
					\noindent\rule{13cm}{0.4pt} 
%					School: Loyola School, Thiruvananthapuram \\
%					12th ISC Marks: 88\% \hfill 
%					10th ICSE Marks: 91.2\% \\
%					\noindent\rule{13cm}{0.4pt} 


				
%\section{Research \\ Interests} 
%{\bf Logic and Automata Theory:} The connections between automata theory and logics for trees and words and its use in Verification. \\
%{\bf Algebraic Automata Theory:} Studying algebraic characterization for different logics over finite as well as infinite words. \\
%{\bf Descriptive Complexity:} Looking at complexity theory from the perspective of Logic. Particularly interested in circuit complexity. \\
%{\bf Interdisciplinary Research:} Modelling water discharge in rivers is an ongoing work with earth scientists. Along with Dr. Kumar Gaurav (Assistant Prof. in IISER, Bhopal) has got funding from Ministry of Earth Sciences (MoES).

\section{Research \\ Project}
{\bf Ministry of Earth Science (MoES)} sanctioned project on ``\emph{Remote sensing based method for detecting water discharge in the Ganga and Brahmaputra rivers}". 
%This is a joint project with Dr. Gaurav Kumar (Earth scientist, IISER Bhopal). As part of the MoES funded project we developed a tool to use satellite imagery to measure water discharge in rivers. The tool automatically identifies a river from satellite images and then measures the width of the river at various points. This tool was then developed into a web based portal.

%\section{Student \\ Guidance}
%Prince Mathew (JRF) in year 2018 for the MoES project. \\
%					\noindent\rule{13cm}{0.4pt} 
%					
\section{PhD \\ Guidance}
Ajay Kumar (JRF) from August 2018 \\
Prince Mathew (JRF) from January 2019

 
\section{Publications}
{\bf 1.} M. Bojanczyk, L. Daviaud, B. Guillon, V. Penelle, A. V. Sreejith ``\emph{Undecidability of a weak version of MSO+U}", Logical Methods in Computer Science (LMCS), volume 16:1, 2020. \\
{\bf 2.} Bharat Adsul, Saptarshi Sarkar, A. V. Sreejith, ``\emph{Block products for algebras over countable words and applications to logic}", Logic in Computer Science (LICS), pages 1-13, 2019. \\
{\bf 3.} Kamal Lodaya, A. V. Sreejith, ``\emph{Two variable first order logic with counting quantifiers: complexity results}'', Developments in Language Theory (DLT), LNCS 10396, pages 260 - 271, 2017. \\
{\bf 4.} Amaldev Manuel, A. V. Sreejith, ``\emph{Two variable logic over countable linear orderings}", Mathematical Foundations of Computer Science (MFCS), LIPIcs 58, pages 66:1-66:13, 2016. \\
{\bf 5.} Thomas Colcombet, A. V. Sreejith, ``\emph{Limited Set quantifiers over Countable Linear Orderings}", Proceedings of the Automata, Languages, and Programming - 42nd International Colloquium (ICALP), volume 9135 of LNCS, pages 146-158, 2015. \\
{\bf 6.} Kamal Lodaya, A. V. Sreejith, ``\emph{Counting quantifiers and linear arithmetic on word models}" with Kamal Lodaya in 14th Asian Logic Conference (ALC), 2014. \\
{\bf 7.} V. Arvind, S. Raja, A. V. Sreejith, ``\emph{On lower bounds for multiplicative circuits and linear circuits in noncommutative domains}", 9th International Computer Science Symposium in Russia (CSR), volume 8476 of LNCS, pages 65-76, 2014.\\


%\section{Unpublished Manuscripts}
%{\bf 1.} A. V. Sreejith, ``\emph{Unary counting and Exponentiation function}''.\\
%{\bf 1.} A Baskar, A. V. Sreejith, Ramanathan S. Thinniyam, ``\emph{Modulo Quantifiers over Functional Vocabularies Extending Addition}'', arXiv preprint arXiv:1705.00290 \\
%{\bf 3.} Bruno Guillon, Laure Daviaud, Miko\l{}aj Boja\'nczyk, Vincent Penelle, A. V. Sreejith,``\emph{Undecidability of MSO+ultimately periodic}''. \\
%{\bf 4.} Miko\l{}aj Boja\'nczyk, A. V. Sreejith, ``\emph{Universal Trees and decidability}". \\
%{\bf 5.} Thomas Colcombet, A. V. Sreejith, ``\emph{Logics over Countable Linear Orderings}". \\
%{\bf 6.} Andreas Krebs, A. V. Sreejith, ``\emph{Generalized first order formulas over (N,+)}". \\
%{\bf 7.} Kamal Lodaya, A. V. Sreejith, ``\emph{Modulo counting logics}". \\
%{\bf 8.} Gaurav K., A. V. Sreejith, Devauchelle O., Sinha R., Metivier F., ``\emph{Remote sensing to estimate formative discharge of the Himalayan foreland rivers}".

%\section{Industrial Visits}
%Visited Akash Lal in Microsoft Research (MSR) \hfill 2018 \\
%Visited Tata Research and Development Center (TRDDC), Pune: I also gave a talk titled, ``\emph{Counting problems in propositional logic}". \hfill 2018
%
%\section{Academic Visits}
%Prof. Kamal Lodaya, IMSc, Chennai \hfill 2018 \\ 					\noindent\rule{13cm}{0.4pt} \\
%Prof. Bharat Adsul, IIT Bombay \hfill 2018 \\					\noindent\rule{13cm}{0.4pt} \\
%Prof. Vincent Penelle, LABRI-University of Bordeaux \hfill 2018 				\noindent\rule{13cm}{0.4pt} \\
%Prof. Dr. Arkadev Chattopadhaya, Tata Institute of Fundamental \hfill March 2016 \\
% Research (TIFR), Mumbai \\
% 					\noindent\rule{13cm}{0.4pt} \\
%Prof. Klaus-J\"orn Lange,  \hfill October 2015\\
%					Wilhelm-Schickard-Institut f\"ur Informatik, \hfill and March to August 2012\\
%					University of T\"ubingen, Germany \hfill and November 2011\\							
%					\noindent\rule{13cm}{0.4pt} 

%\section{Some Talks}
%{$\bullet$} ``\emph{MSO extended with bounded and periodic predicates}'', invited speaker at Automata, Concurrency and Timed Systems (ACTS) workshop, CMI Feb-17. \\
%{$\bullet$} Invited to give an one week course (five lectures of one hour each), on ``\emph{Descriptive complexity theory: An introduction}" from 21-26 March, 2016 for Indian School on Logic and its Applications (ISLA),  PSG college of Technology, Coimbatore. \\
%{$\bullet$} ``\emph{Two-Variable logic over countable linear orderings}", Mathematical foundations of computer science, Krakow, August 2016 \\
%{$\bullet$}``\emph{Limited set quantifiers over countable linear orderings}" was given in the following places: ICALP, Kyoto, July 2015; Highlights conference, Prague, September 2015; University of T\"ubingen, October 2015; Chennai Theory Day, April 2016. \\
%{$\bullet$} ``\emph{Non-definability of languages by generalized first-order formulas over ($\mathbb{N},+$)}" was given in the following places: University of Stuttgart, Germany, April 2012; LICS, Dubrovnik, Croatia, June 2012; Highlights conference, Paris, September 2014; Dagstuhl, Saarbrucken, Germany, October 2016. \\
%%{$\bullet$} ``\emph{An introduction to theoretical computer science}": A general audience talk in Maison de L'Inde, Paris, February 2015. \\
%{$\bullet$} ``\emph{Power of multiplication}", Indian Statistical Institute, Chennai, December 2013. \\
%{$\bullet$} ``\emph{Expressive completeness for LTL with modulo counting and group operators}", 7th Methods for Modalities workshop, November 2011, Spain.  \\
%{$\bullet$}``\emph{Logics with counting quantifiers}",  University of T\"ubingen, November 2011.  \\
%%{$\bullet$}``\emph{Deciding presburger arithmetic}", Formal Methods Update Meeting, July 2011, VIT University, Vellore, India.  \\
%%{$\bullet$} ``\emph{LTL can be more succinct}", 8th International Symposium on Automated Technology for Verification and Analysis, September 2010, Singapore. \\
%%{$\bullet$} ``\emph{Automata, treewidth and regular expressions, Formal Methods Update Meeting}", July 2010, Dhirubhai Ambani Institute of Information and Communication Technology, Gandhinagar, India.
%%{$\bullet$} ``\emph{Coding tiling problem by Logic}", Institute Seminar Week, IMSc, March 2010.


\section{Teaching Experience}
					``\emph{Data structures and algorithms}'' \hfill from January 2020 \\
					Undergraduate students in Indian Institute of Technology Goa \\
					\noindent\rule{13cm}{0.4pt}
					``\emph{Combinatorial optimization}'' \hfill from January 2020 \\
					Graduate and Undergraduate students in IIT Goa \\
					\noindent \rule{13cm}{0.4pt}
					``\emph{Logic for Computer Science}'' \hfill from July 2019 \\
					Undergraduate students in Indian Institute of Technology Goa \hfill to December 2019  \\
					\noindent\rule{13cm}{0.4pt}
					``\emph{Model checking and software verification}'' \hfill from July 2019 \\
					Graduate and Undergraduate students in IIT Goa \hfill to December 2019   \\
					\noindent\rule{13cm}{0.4pt}
					``\emph{Advanced algorithms}'' \hfill January 2019 \\
					Undergraduate students in Indian Institute of Technology Goa \hfill to April 2019 \\
					\noindent\rule{13cm}{0.4pt}
					``\emph{Topics in Logic and Automata theory}'' \hfill January 2019 \\
					Graduate students in Indian Institute of Technology Goa \hfill to April 2019 \\
					\noindent\rule{13cm}{0.4pt}
					``\emph{Logic for computer science}'' \hfill August 2018 \\
					Undergraduate students in Indian Institute of Technology Goa \hfill to December 2018 \\
					\noindent\rule{13cm}{0.4pt}
%					``\emph{Software systems lab}''   \hfill August 2018 \\
%					Undergraduate students in Indian Institute of Technology Goa \hfill to December 2018 \\
%					\noindent\rule{13cm}{0.4pt}					
					``\emph{Computer Programming course and lab}'' for  \hfill Summer course 2018 \\
					Undergraduate students in Indian Institute of Technology Goa \hfill  \\
					\noindent\rule{13cm}{0.4pt}					
					``\emph{Computer Networks course and lab}'' for  \hfill January 2018 \\
					Undergraduate students in Indian Institute of Technology Goa \hfill to May 2018 \\
					\noindent\rule{13cm}{0.4pt}
					Course titled: ``\emph{Verification}" for Graduate  \hfill March 2012  \\
					and Undergraduate students in Wilhelm-Schickard-Institut \hfill to August 2012\\ 
					f\"ur Informatik, University of T\"ubingen, Germany \\
					\noindent\rule{13cm}{0.4pt} \\
%					``\emph{Algebraic Automata Theory}'' for a month. \hfill  \\
%					 Graduate and Undergraduate students in \hfill  August 2016 \\
%					 Chennai Mathematical Institute (CMI), Chennai \\
%					\noindent\rule{13cm}{0.4pt}

%\section{Administrative Duties}
%Heading the Center for Information Technology Services (CITS) in IIT Goa \\
% 					\noindent\rule{13cm}{0.4pt} \\
%Member of the Senate Under-graduate Committee (SUGC). \\
% 					\noindent\rule{13cm}{0.4pt} \\
%Member of the Senate Students Advisory Committee (SSAC)\\
% 					\noindent\rule{13cm}{0.4pt} \\
%Member of the website and intranet maintaining committee. 										

			
%\section{References}
%			Name: 		Prof.  Miko\l{}aj Boja\'nczyk \\
%			Institute: 	Institute of Informatics, University of Warsaw, MIMUW, \\
%			Banacha 2, 02-097 Warszawa, Poland \\
%			\emph{E-mail:} {bojan@mimuw.edu.pl} \\
%			Home Page: \href{http://www.mimuw.edu.pl/~bojan/}{\tt{http://www.mimuw.edu.pl/$\sim$bojan/}}\\
%			\noindent\rule{13cm}{0.4pt} \\					
%			Name: 		Prof. Kamal Lodaya (Thesis adviser) \\
%			Institute: 	The Institute of Mathematical Sciences (IMSc), \\
%			IV Cross Road, CIT Campus, Taramani, Chennai - 600 113	\\
%			\emph{E-mail:} {kamal@imsc.res.in} \\
%			Home Page: \href{http://www.imsc.res.in/~kamal/}{\tt{http://www.imsc.res.in/$\sim$kamal/}} \\
%			\noindent\rule{13cm}{0.4pt} \\				
%			Name: 		Dr. Andreas Krebs \\
%			Institute: 	Universit\"at T\"ubingen, \\
%			Wilhelm-Schickard-Institut f\"ur Informatik, Sand 13, D-72076, T\"ubingen, Germany \\
%			\emph{E-mail:} {krebs[at]informatik.uni-tuebingen.de} \\
%			Home Page: \href{http://www.uni-tuebingen.de/en/faculties/faculty-of-science/departments/computer-science/lehrstuehle/theoretische-informatik/home/mitarbeiter/dr-andreas-krebs.html}{\tt{www.uni-tuebingen.de/en/faculties/faculty-of-science/departments/computer}}\\
%			\href{http://www.uni-tuebingen.de/en/faculties/faculty-of-science/departments/computer-science/lehrstuehle/theoretische-informatik/home/mitarbeiter/dr-andreas-krebs.html}{\tt{-science/lehrstuehle/theoretische-informatik/home/mitarbeiter/dr-andreas-krebs.html}} 
\end{resume}
\end{document}




