% LaTeX resume using res.cls
\documentclass[margin]{res}
%\usepackage{helvetica} % uses helvetica postscript font (download helvetica.sty)
%\usepackage{newcent}   % uses new century schoolbook postscript font 
\setlength{\textwidth}{5.1in} % set width of text portion
\usepackage{etaremune}

\begin{document}

% Center the name over the entire width of resume:
 \moveleft.5\hoffset\centerline{\large\bf Sreejith A V}
% Draw a horizontal line the whole width of resume:
 \moveleft\hoffset\vbox{\hrule width\resumewidth height 1pt}\smallskip
% address begins here
% Again, the address lines must be centered over entire width of resume:
 \moveleft.5\hoffset\centerline{University of Warsaw,}
 \moveleft.5\hoffset\centerline{MIMUW, Banacha 2, 02-097 Warszawa, Poland}
 \moveleft.5\hoffset\centerline{E-mail:sreejithav@mimuw.edu.pl}
 \moveleft.5\hoffset\centerline{Home page:www.mimuw.edu.pl/$\sim$sreejithav/}
 

\begin{resume}
 
\section{Broad Subject}  Computer science, Theoretical computer science
\section{Area of Research} Logic, Automata theory, Algebraic automata theory, Descriptive complexity

\section{Awards}
\emph{ACM India Doctoral Dissertation Award 2014: Honourable mention} for doctoral dissertation, ``Regular quantifiers in Logics".

\section{Research Experience} PostDoc at Institute of Informatics, \hfill September 2016 \\
					University of Warsaw, Poland \hfill to at present \\
					\noindent\rule{13cm}{0.4pt} \\
					Visiting Faculty at 	\hfill December 2015 \\
					Chennai Mathematical Institute (CMI)  \hfill to August 2016 \\
					\noindent\rule{13cm}{0.4pt} \\
					PostDoc at Laboratoire d{'}Informatique Algorithmique: \hfill April 2014\\
					Fondements et Applications (LIAFA),	Paris, France \hfill September 2015 \\
					\noindent\rule{13cm}{0.4pt} \\
					PostDoc at Tata Institute of Fundamental Research (TIFR),  \hfill September 2013 \\
					Mumbai	\hfill	to March 2014
					\noindent\rule{13cm}{0.4pt} \\					
					Visited Dr. Klaus-J\"orn Lange, Wilhelm-Schickard-Institut \hfill March 2012 to\\
					f\"ur Informatik, University of T\"ubingen, Germany \hfill August 2012 
					\noindent\rule{13cm}{0.4pt} \\
					PhD in Theoretical Computer Science at \\
					Institute of Mathematical Sciences(IMSc), Chennai \\
					Advisor: Prof. Kamal Lodaya \hfill August 2007 to \\
					Thesis: Regular Quantifiers in Logics \hfill September 2013 \\
					\noindent\rule{13cm}{0.4pt} 
									
\section{Teaching Experience}
					One semester course titled ``\emph{Verification}" for Graduate \hfill March 2012 to  \\
					and Undergraduate students in \emph{Wilhelm-Schickard-Institut} \hfill August 2012\\ 
					\emph{f\"ur Informatik, University of T\"ubingen, Germany} \\
					\noindent\rule{13cm}{0.4pt} \\
					One week course on ``\emph{Descriptive Complexity theory: An introduction}", \hfill 21 March \\
					 Indian School on Logic and its Applications (ISLA),   \hfill 26 March, 2016 \\
					 PSG college of Technology, Coimbatore \\
					\noindent\rule{13cm}{0.4pt} 

\section{Education} 
					MTech in Computer Science \hfill August 2005 to \\
					Institute: Indian Institute of Technology (IIT), Madras \hfill May 2007 \\
					Project guide: Prof. C. Chandra Sekhar \\
					\noindent\rule{13cm}{0.4pt} \\
					BTech in Computer Science \hfill August 2000 to \\
					Institute: College of Engineering, Thiruvananthapuram \hfill June 2004 \\
					\noindent\rule{13cm}{0.4pt} 

\section{Industry Experience}
Software Engineer, Infosys Technologies, India \hfill August 2004-July 2005
 
\section{Publications}
{\bf 7.} Amaldev Manuel, A. V. Sreejith, ``Two variable logic over countable linear orderings", Accepted in Proceedings of Mathematical Foundations for Computer Science (MFCS), 2016 \\
{\bf 6.} Thomas Colcombet, A. V. Sreejith, ``Limited Set quantifiers over Countable Linear Orderings", Proceedings of the Automata, Languages, and Programming
- 42nd International Colloquium (ICALP), 2015, 146-158 \\
{\bf 5.} Kamal Lodaya, A. V. Sreejith, ``Counting quantifiers and linear arithmetic on word models" with Kamal Lodaya in 14th Asian Logic Conference (ALC), 2014. \\
{\bf 4.} V Arvind, S Raja, A V Sreejith, ``On lower bounds for multiplicative circuits and linear circuits in noncommutative domains", 9th International Computer Science Symposium in Russia (CSR), 2014, 65-76\\
{\bf 3.} Andreas Krebs, A V Sreejith, ``Non-definability of Languages by Generalized First-order Formulas over (N, +)", Proceedings of the 27th Annual IEEE Symposium on Logic in Computer Science (LICS), 2012, 451-460.\\
{\bf 2.} A V Sreejith, ``Expressive Completeness for LTL With Modulo Counting and Group Quantifiers", M4M, 2011, Vol-278, 201-214.\\
{\bf 1.} Kamal Lodaya, A V Sreejith, ``LTL can be more succinct", Proceedings of 8th International Symposium on Automated Technology for Verification and Analysis (ATVA), 2010, LNCS 6252, 21-24.

\section{Academic Visits}
Dr. Arkadev Chattopadhaya, Tata Institute of Fundamental \hfill March 2016 \\
 Research (TIFR), Mumbai \\
 					\noindent\rule{13cm}{0.4pt} \\
Prof. Klaus-J\"orn Lange, Wilhelm-Schickard-Institut f\"ur Informatik, \hfill October 2015\\
					University of T\"ubingen, Germany \\							
					\noindent\rule{13cm}{0.4pt} \\
Prof. Klaus-J\"orn Lange, Wilhelm-Schickard-Institut f\"ur Informatik, \hfill March 2012 to\\
					University of T\"ubingen, Germany \hfill August 2012 \\
					\noindent\rule{13cm}{0.4pt} \\
Prof. Klaus-J\"orn Lange, Wilhelm-Schickard-Institut f\"ur Informatik, \hfill November 2011\\
					University of T\"ubingen, Germany \\					
					\noindent\rule{13cm}{0.4pt} \\

\section{References}
			Name: 		Prof.  Miko\l{}aj Boja\'nczyk \\
			Institute: 	Institute of Informatics, University of Warsaw, Poland\\
			\emph{E-mail:} {bojan@mimuw.edu.pl} \\
			Home Page: www.mimuw.edu.pl/$\sim$bojan/\\
			\noindent\rule{13cm}{0.4pt} \\					
			Name: 		Prof. Kamal Lodaya (Thesis adviser) \\
			Institute: 	The Institute of Mathematical Sciences (IMSc) \\
			CIT Campus, Taramani, Chennai	\\
			\emph{E-mail:} {kamal@imsc.res.in} \\
			Home Page: www.imsc.res.in/$\sim$kamal/ \\
			\noindent\rule{13cm}{0.4pt} \\					
			Name: 		Prof. V. Arvind \\
			Institute: 	The Institute of Mathematical Sciences (IMSc) \\
			CIT Campus, Taramani, Chennai	\\
			\emph{E-mail:} {arvind@imsc.res.in} \\
			Home Page: www.imsc.res.in/$\sim$arvind/ \\
			\noindent\rule{13cm}{0.4pt}	
			
\end{resume}
\end{document}




