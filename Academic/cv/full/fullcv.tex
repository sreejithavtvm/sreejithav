% LaTeX resume using res.cls
\documentclass[margin]{res}
%\usepackage{helvetica} % uses helvetica postscript font (download helvetica.sty)
%\usepackage{newcent}   % uses new century schoolbook postscript font 
\setlength{\textwidth}{5.1in} % set width of text portion
\usepackage{etaremune}

\begin{document}

% Center the name over the entire width of resume:
 \moveleft.5\hoffset\centerline{{\Large \bf Sreejith A V} \hfill \emph{ sreejithav@mimuw.edu.pl}}
% Draw a horizontal line the whole width of resume:
 \moveleft\hoffset\vbox{\hrule width\resumewidth height 1pt}\smallskip
% address begins here
% Again, the address lines must be centered over entire width of resume:
\moveleft.5\hoffset\centerline{Indian Institute of Technology Goa}
\moveleft.5\hoffset\centerline{GEC campus, Farmagudi, Goa - 403401}
\moveleft.5\hoffset\centerline{E-mail: sreejithav@iitgoa.ac.in}
\moveleft.5\hoffset\centerline{ \href{http://www.iitgoa.ac.in/~sreejithav/} {\tt{www.iitgoa.ac.in/$\sim$sreejithav/}}}
\moveleft.5\hoffset\centerline{Phone: +918939601670}

 
\begin{resume}
 
\section{Broad Subject}  Computer science, Theoretical computer science
\section{Area of Research} Logic, Automata theory, Algebraic automata theory, Descriptive complexity

%\section{Personal Information}
%Date of Birth: 21 March, 1983 \\
%Gender: Male \\
%Nationality: Indian

\section{Research Experience} 
					Assistant Professor at \\
					Indian Institute of Technology (IIT) Goa \hfill October 2017 to at present \\
					\noindent\rule{13cm}{0.4pt} \\
					Postdoctoral Researcher at \hfill  \hfill September 2016 to January 2017 \\
					Institute of Informatics, University of Warsaw \hfill July 2017 to September 2017 \\
					Mentor: Prof. Miko\l{}aj Boja\'nczyk \\
					\noindent\rule{13cm}{0.4pt} \\
					Visiting Faculty at 	\hfill December 2015 to August 2016 \\
					Chennai Mathematical Institute (CMI)  \hfill to February 2017 to June 2017 \\
					\noindent\rule{13cm}{0.4pt} \\
					Postdoctoral Researcher  at Laboratoire d{'}Informatique Algorithmique: \hfill April 2014\\
					Fondements et Applications (LIAFA),	Paris, France \hfill to September 2015 \\
					Mentor: Prof. Thomas Colcombet \\
					\noindent\rule{13cm}{0.4pt} \\
					Postdoctoral Researcher  at Tata Institute of  Fundamental \hfill September 2013 \\
					Research (TIFR), Mumbai	\hfill	to March 2014
					\noindent\rule{13cm}{0.4pt} \\					
					Research Position with Dr. Klaus-J\"orn Lange, Wilhelm-Schickard \hfill March 2012\\
					-Institut f\"ur Informatik, University of T\"ubingen, Germany \hfill to August 2012 
					\noindent\rule{13cm}{0.4pt} \\
					PhD in Mathematics at \hfill August 2007\\
					Institute of Mathematical Sciences (IMSc), Chennai \hfill to September 2013\\
					Advisor: Prof. Kamal Lodaya  \\
					Thesis: Regular Quantifiers in Logics  \\
					\noindent\rule{13cm}{0.4pt} 

									
\section{Teaching Experience}
					Took a one semester course titled ``\emph{Verification}" for Graduate \hfill March 2012 to  \\
					and Undergraduate students in \emph{Wilhelm-Schickard-Institut} \hfill August 2012\\ 
					\emph{f\"ur Informatik, University of T\"ubingen, Germany} \\
					\noindent\rule{13cm}{0.4pt} \\
					One week course on ``\emph{Descriptive Complexity theory: An introduction}", \hfill 21 March \\
					 Indian School on Logic and its Applications (ISLA),   \hfill to 26 March, 2016 \\
					 PSG college of Technology, Coimbatore \\
					\noindent\rule{13cm}{0.4pt} 

\section{Industry Experience}
Software Engineer, Infosys Technologies, India \hfill August 2004-July 2005

\section{Education} 
					MTech in Computer Science \hfill August 2005 to \\
					Institute: Indian Institute of Technology (IIT), Madras \hfill May 2007 \\
					Project guide: Prof. C. Chandra Sekhar \\
					CGPA: 8.86 \\
					\noindent\rule{13cm}{0.4pt} \\
					BTech in Computer Science \hfill August 2000 to \\
					Institute: College of Engineering, Thiruvananthapuram \hfill June 2004 \\
					Marks: 73\% \\
					\noindent\rule{13cm}{0.4pt} 
					School: Loyola School, Thiruvananthapuram \\
					12th ISC Marks: 88\% \\
					10th ICSE Marks: 91.2\% \\
					\noindent\rule{13cm}{0.4pt} 
					
					
\section{Awards}
\emph{ACM India Doctoral Dissertation Award 2014: Honourable mention} for doctoral dissertation, ``Regular quantifiers in Logics".
 
\section{Publications}
{\bf 9.} M. Bojanczyk, L. Daviaud, B. Guillon, V. Penelle, A. V. Sreejith ``\emph{Undecidability of a weak version of MSO+U}", Logical Methods in Computer Science (LMCS), 2019 (accepted).
{\bf 8.} Bharat Adsul, Saptarshi Sarkar, A. V. Sreejith, ``\emph{Block products for algebras over countable words and applications to logic}", Logic in Computer Science (LICS), 2019 (accepted).
{\bf 7.} Amaldev Manuel, A. V. Sreejith, ``Two variable logic over countable linear orderings", Accepted in Proceedings of Mathematical Foundations of Computer Science (MFCS), 2016 \\
Impact Factor:0.48 (top 52.25\%), Core Rank:A, ERA Rank:A, MS Field rating:42 \\
{\bf 6.} Thomas Colcombet, A. V. Sreejith, ``Limited Set quantifiers over Countable Linear Orderings", Proceedings of the Automata, Languages, and Programming - 42nd International Colloquium (ICALP), 2015, 146-158 \\
Impact Factor:0.97 (top 24.65\%), Core Rank:A, ERA Rank:A, MS Field rating:75\\
{\bf 5.} Kamal Lodaya, A. V. Sreejith, ``Counting quantifiers and linear arithmetic on word models" with Kamal Lodaya in 14th Asian Logic Conference (ALC), 2014. \\
{\bf 4.} V. Arvind, S. Raja, A. V. Sreejith, ``On lower bounds for multiplicative circuits and linear circuits in noncommutative domains", 9th International Computer Science Symposium in Russia (CSR), 2014, 65-76\\
Impact Factor:0.51, Core Rank:C, ERA Rank:C, MS Field rating:11 \\
{\bf 3.} Andreas Krebs, A. V. Sreejith, ``Non-definability of Languages by Generalized First-order Formulas over (N, +)", Proceedings of the 27th Annual IEEE Symposium on Logic in Computer Science (LICS), 2012, 451-460.\\
Impact Factor:1.79 (top 4.50\%), Core Rank:A*, ERA Rank:A, MS Field rating:79 \\
{\bf 2.} A. V. Sreejith, ``Expressive Completeness for LTL With Modulo Counting and Group Quantifiers", Method for Modalities (M4M), 2011, ENTCS Vol-278, 201-214.\\
{\bf 1.} Kamal Lodaya, A. V. Sreejith, ``LTL can be more succinct", Proceedings of 8th International Symposium on Automated Technology for Verification and Analysis (ATVA), 2010, LNCS 6252, 21-24. \\
Impact Factor:0.30, Core Rank:A, ERA Rank:A, MS Field rating:14\\


\section{Unpublished manuscripts}
{\bf 4.} Thomas Colcombet, A. V. Sreejith, ``Logics over Countable Linear Orderings". \\
{\bf 3.} Andreas Krebs, A. V. Sreejith, ``Generalized first order formulas over (N,+)". \\
{\bf 2.} Kamal Lodaya, A. V. Sreejith, ``Two variable logics with counting quantifiers". \\
{\bf 1.} Gaurav. K, A. V. Sreejith, Devauchelle. O, Sinha. R, Metivier. F, ``Remote sensing to estimate formative discharge of the Himalayan foreland rivers".


\section{Academic Visits}
Dr. Arkadev Chattopadhaya, Tata Institute of Fundamental \hfill March 2016 \\
 Research (TIFR), Mumbai \\
 					\noindent\rule{13cm}{0.4pt} \\
Prof. Klaus-J\"orn Lange, Wilhelm-Schickard-Institut f\"ur Informatik, \hfill October 2015\\
					University of T\"ubingen, Germany \\							
					\noindent\rule{13cm}{0.4pt} \\
Prof. Klaus-J\"orn Lange, Wilhelm-Schickard-Institut f\"ur Informatik, \hfill March 2012 to\\
					University of T\"ubingen, Germany \hfill August 2012 \\
					\noindent\rule{13cm}{0.4pt} \\
Prof. Klaus-J\"orn Lange, Wilhelm-Schickard-Institut f\"ur Informatik, \hfill November 2011\\
					University of T\"ubingen, Germany \\					
					\noindent\rule{13cm}{0.4pt} \\

\section{Talks}
\begin{itemize}
\item Countable linear orderings, University of T\"ubingen, October 2016, Germany. 
\item Countable linear orderings, Chennai Theory day, 
\item 6 day Invited course on Descriptive complexity, ISLA, Coimbatore, 2016
\item Countable linear orderings, Icalp, kyoto, japan,
\item Countable linear orderings, Highlights, czech republic
\item Non-definability of Languages by Generalized First-order Formulas over (N, +), Dagstuhl, October 2016, Germany. 
\item Non-definability of Languages by Generalized First-order Formulas over (N, +), Logic in Computer Science (LICs), June 2012, Dubrovnik, Croatia. 
\item Non-definability of Languages by Generalized First-order Formulas over (N, +), Institut f\"ubr Formale Methoden der Informatik,  April 2012, Stuttgart, Germany. 
\item Expressive completeness for LTL with modulo counting and group operators Talk given at 7th Methods for Modalities workshop, November 2011, Spain. 
\item Logics with Counting Quantifiers,  University of T\"ubngen, November 2011, Germany. 
\item Deciding Presburger Arithmetic, Formal Methods Update Meeting, July 2011, VIT University, Vellore, India. 
\item LTL can be more succinct, 8th International Symposium on Automated Technology for Verification and Analysis, September 2010, Singapore. 
\item Automata, treewidth and regular expressions, Formal Methods Update Meeting, July 2010, Dhirubhai Ambani Institute of Information and Communication Technology, Gandhinagar, India. 
\item Coding Tiling Problem by Logic, Institute Seminar Week, March 2010, IMSc, Chennai, India. 
\end{itemize}

\section{References}
			Name: 		Prof. Thomas Colcombet \\
			Institute: 	Laboratoire d{'}Informatique Algorithmique:  \\
			Fondements et Applications (LIAFA), Paris, France	\\
			\emph{E-mail:} {thomas.colcombet@liafa.univ-paris-diderot.fr} \\
			Home Page: http://www.liafa.univ-paris-diderot.fr/$\sim$colcombe/ \\
			\noindent\rule{13cm}{0.4pt} \\					
			Name: 		Prof. Kamal Lodaya (Thesis adviser) \\
			Institute: 	The Institute of Mathematical Sciences (IMSc) \\
			CIT Campus, Taramani, Chennai	\\
			\emph{E-mail:} {kamal@imsc.res.in} \\
			Home Page: http://www.imsc.res.in/$\sim$kamal/ \\
			\noindent\rule{13cm}{0.4pt} \\					
			Name: 		Prof. V. Arvind \\
			Institute: 	The Institute of Mathematical Sciences (IMSc) \\
			CIT Campus, Taramani, Chennai	\\
			\emph{E-mail:} {arvind@imsc.res.in} \\
			Home Page: http://www.imsc.res.in/$\sim$arvind/ \\
			\noindent\rule{13cm}{0.4pt} \\					
			Name: 		Prof. R. Ramanujam \\
			Institute: 	The Institute of Mathematical Sciences (IMSc) \\
			CIT Campus, Taramani, Chennai	\\
			\emph{E-mail:} {jam@imsc.res.in} \\
			Home Page: http://www.imsc.res.in/$\sim$jam/ \\
			\noindent\rule{13cm}{0.4pt} \\					
			Name: 		Dr. Arkadev Chattopadhaya \\
			Institute: 	Tata Institute of Fundamental Research (TIFR), \\
			Homi Bhabha Road, Colaba, Mumbai	\\
			\emph{E-mail:} {c.arkadev@tifr.res.in} \\
			Home Page: http://www.imsc.res.in/$\sim$arkadev/ \\
			\noindent\rule{13cm}{0.4pt} \\					
			
\end{resume}
\end{document}




