 % LaTeX resume using res.cls
\documentclass[margin]{res}
\usepackage{amsfonts}

%\usepackage{helvetica} % uses helvetica postscript font (download helvetica.sty)
%\usepackage{newcent}   % uses new century schoolbook postscript font 
\setlength{\textwidth}{5.1in} % set width of text portion
\usepackage{etaremune}
\usepackage{hyperref}
\hypersetup{
hidelinks
}

\begin{document}
% Center the name over the entire width of resume:
\moveleft.5\hoffset\centerline{\Large \bf A. V. Sreejith} 
% Draw a horizontal line the whole width of resume:
\moveleft\hoffset\vbox{\hrule width\resumewidth height 1pt}\smallskip
% address begins here
% Again, the address lines must be centered over entire width of resume:
\moveleft.5\hoffset\centerline{Associate Professor, School of Mathematics and Computer Science}
\moveleft.5\hoffset\centerline{Indian Institute of Technology Goa, GEC campus, Farmagudi, Goa - 403401}
\moveleft.5\hoffset\centerline{E-mail: sreejithav@iitgoa.ac.in ~~~~ \href{http://www.iitgoa.ac.in/~sreejithav/} {\tt{www.iitgoa.ac.in/$\sim$sreejithav/}}}
%\moveleft.5\hoffset\centerline{ \href{http://www.iitgoa.ac.in/~sreejithav/} {\tt{www.iitgoa.ac.in/$\sim$sreejithav/}}}
%\moveleft.5\hoffset\centerline{Phone: +918939601670}

 
\begin{resume}
 
\section{Broad Subject}  Computer Science, Theoretical Computer Science
%, Formal methods and Verification
%\section{Personal Information}
%Date of birth: March 21, 1983 \hfill Citizenship: Indian
%\begin{tabular}{ll}
%Date of Birth: & March 21, 1983 \\
%Gender: & Male \\
%Citizenship: & Indian \\ 
%Permanent Address: & TC 14/1328, Kochu Madom, Vazhuthacaud, \\
% & Thiruvananthapuram, Kerala-695014
%\end{tabular}

\section{Research Experience} 
					Associate Professor, School of Mathematics and Computer Science, \\
					Indian Institute of Technology Goa  \hfill from October 2017 \\
					(also associated with School of Interdisciplinary Life Sciences from November 2020) \\
					\noindent\rule{13cm}{0.4pt} \\
					Postdoctoral Researcher at  \\
					Institute of Informatics, University of Warsaw \\
					\noindent\rule{13cm}{0.4pt} \\
					Postdoctoral Researcher at 	\\
					Chennai Mathematical Institute (CMI)  \\
					\noindent\rule{13cm}{0.4pt} \\
					Postdoctoral Researcher  at Laboratoire d{'}Informatique Algorithmique: \\
					Fondements et Applications (LIAFA),	Paris, France  \\
					\noindent\rule{13cm}{0.4pt} \\
					Postdoctoral Researcher  at Tata Institute of  Fundamental  Research (TIFR), Mumbai \\
					\noindent\rule{13cm}{0.4pt} \\					
					Research Position with Dr. Klaus-J\"orn Lange, Wilhelm-Schickard \\
					-Institut f\"ur Informatik, University of T\"ubingen, Germany \\
					\noindent\rule{13cm}{0.4pt} \\
					PhD in Mathematics at \\
					Institute of Mathematical Sciences (IMSc), Chennai  \\
					Advisor: Prof. Kamal Lodaya   \\
					Thesis: Regular Quantifiers in Logics  \\
					\noindent\rule{13cm}{0.4pt} 

\section{Industry Experience}
Software Engineer, Infosys Technologies, India

%\newpage
\section{Education} 
					MTech in Computer Science \\
					Institute: Indian Institute of Technology (IIT), Madras \\
					%Project: Automatic Raga classification \\
					%Project guide: Prof. C. Chandra Sekhar \hfill CGPA: 8.86 \\
					\noindent\rule{13cm}{0.4pt} \\
					BTech in Computer Science \\
					Institute: College of Engineering, Thiruvananthapuram  \\
					\noindent\rule{13cm}{0.4pt} 
%					School: Loyola School, Thiruvananthapuram \\
%					12th ISC Marks: 88\% \hfill 
%					10th ICSE Marks: 91.2\% \\
%					\noindent\rule{13cm}{0.4pt} 
	
\section{Awards}
\href{http://awards.acm.org/award_winners/sreejith_9923861.cfm}{{\emph{ACM India Doctoral Dissertation Award 2014: Honourable mention}}} for doctoral dissertation, ``Regular Quantifiers in Logics". The award is given by the Association for computing machinery (ACM), India. \\
%The award is for the second best doctoral dissertation in computer science in India for the year 2014. 
%{$\bullet$} First prize in programming contest conducted by Cochin University in 2002. \\
%\emph{Regional Mathematical Olympiad (RMO)} in 10th standard.
				
					
\section{Ph.D Guidance}
%Ajay Kumar from August 2018: Working on Learning \\
Prince Mathew (ongoing): Working on weighted one counter automata

\section{Other activities}
$\bullet$ Co-ordinator - the Center for Information Technology Services in IIT Goa \\
$\bullet$ Member of the expert committee for Goa startup promotion cell


\section{Teaching Experience}
					{\bf Indian Institute of Technology Goa } \\
					\noindent\rule{13cm}{0.4pt}
					\begin{enumerate}
					\item {Data structures and algorithms} (undergraduate course) \hfill Aug-Dec 2021
					\item {Higher dimensional data science} (graduate/undergraduate) \hfill Jan-April 2021
					\item {Data structures and algorithms} (undergraduate course)  \hfill  Aug-Dec 2020 
					%\noindent\rule{13cm}{0.4pt}
					\item {Algorithm Design lab} (post graduate course) \hfill  Aug-Dec 2020
					\item {Data structures and algorithms} (undergraduate course) \hfill  Jan-May 2020 			
					\item {Combinatorial optimization} (graduate/undergraduate course) \hfill  Jan-May 2020 
					
					\item {Logic for Computer Science}  (undergraduate course) \hfill  July-Dec 2019
					\item {Advanced algorithms} (undergraduate course) \hfill Jan-April 2019 					
					\item {Topics in Logic and Automata theory} (graduate course) \hfill Jan-April 2019
					\item {Logic for computer science} (undergraduate course) \hfill Aug-Dec 2018 
					\item {Computer Networks course and lab} (undergraduate course)  \hfill Jan-May 2018 
					\end{enumerate}
					
					{\bf University of T\"ubingen, Germany } \\
					\noindent\rule{13cm}{0.4pt}
					\begin{enumerate}
					\item {Formal Verification}  (undergraduate and graduate course) \hfill March-Aug 2012 
					\end{enumerate}
									

\section{Publications}
\begin{enumerate}
\item Gaurav K., Métivier F., A. V. Sreejith, Sinha R., Kumar A., and Tandon S. K. ``\emph{Coupling threshold theory and satellite-derived channel width to estimate the discharge of Himalayan foreland rivers}'', Earth Surf. Dynam., 9, 47-70, 2021
\item M. Bojanczyk, L. Daviaud, B. Guillon, V. Penelle, A. V. Sreejith ``\emph{Undecidability of a weak version of MSO+U}", LMCS, volume 16:1, 2020.
\item Bharat Adsul, Saptarshi Sarkar, A. V. Sreejith, ``\emph{Block products for algebras over countable words and applications to logic}", LICS, pages 1-13, 2019.
\item Kamal Lodaya, A. V. Sreejith, ``\emph{Two variable first order logic with counting quantifiers: complexity results}'', DLT, LNCS 10396, pages 260 - 271, 2017.
\item Amaldev Manuel, A. V. Sreejith, ``\emph{Two variable logic over countable linear orderings}", MFCS, LIPIcs 58, pages 66:1-66:13, 2016. 
\item Thomas Colcombet, A. V. Sreejith, ``\emph{Limited Set quantifiers over Countable Linear Orderings}", ICALP, volume 9135 of LNCS, pages 146-158, 2015. 
\item Kamal Lodaya, A. V. Sreejith, ``\emph{Counting quantifiers and linear arithmetic on word models}" with Kamal Lodaya in 14th Asian Logic Conference (ALC), 2014.
\item V. Arvind, S. Raja, A. V. Sreejith, ``\emph{On lower bounds for multiplicative circuits and linear circuits in noncommutative domains}", CSR, 65-76, 2014.
\item Andreas Krebs, A. V. Sreejith, ``\emph{Non-definability of Languages by Generalized First-order Formulas over (N, +)}", LICS,  pages 451-460, 2012.
\item A. V. Sreejith, ``\emph{Expressive Completeness for LTL With Modulo Counting and Group Quantifiers}", ENTCS 278, pages 201-214, 2011.
\item Kamal Lodaya, A. V. Sreejith, ``\emph{LTL can be more succinct}", ATVA, 21-24, 2010.
\end{enumerate}



\end{resume}
\end{document}




