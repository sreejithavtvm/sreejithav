% LaTeX resume using res.cls
\documentclass[margin]{res}
%\usepackage{helvetica} % uses helvetica postscript font (download helvetica.sty)
%\usepackage{newcent}   % uses new century schoolbook postscript font 
\setlength{\textwidth}{5.1in} % set width of text portion
\usepackage{etaremune}

\begin{document}

% Center the name over the entire width of resume:
 \moveleft.5\hoffset\centerline{\large\bf Sreejith A V}
% Draw a horizontal line the whole width of resume:
 \moveleft\hoffset\vbox{\hrule width\resumewidth height 1pt}\smallskip
% address begins here
% Again, the address lines must be centered over entire width of resume:
 \moveleft.5\hoffset\centerline{Visiting Faculty,}
 \moveleft.5\hoffset\centerline{Chennai Mathematical Institute(CMI),}
 \moveleft.5\hoffset\centerline{H1, SIPCOT IT Park, Siruseri, Kelambakkam 603103, India}
 \moveleft.5\hoffset\centerline{E-mail:sreejithav@gmail}
 \moveleft.5\hoffset\centerline{Phone:+918939601670} 
 \moveleft.5\hoffset\centerline{Home page: www.cmi.ac.in/$\sim$sreejithav}
 

\begin{resume}
 
\section{Broad Subject}  Computer science, Theoretical computer science
\section{Area of Research} Logic, Automata theory, Algebraic automata theory, Descriptive complexity

\section{Personal Information}
Date of Birth: 21 March, 1983 \\
Gender: Male \\
Nationality: Indian

\section{Research Experience}Visiting Faculty, 	\hfill December 2015 \\
					Chennai Mathematical Institute \hfill to at present \\
					\noindent\rule{13cm}{0.4pt} \\
					PostDoc at Laboratoire d{'}Informatique Algorithmique: \hfill April 2014\\
					Fondements et Applications (LIAFA),	Paris, France \hfill to September 2015 \\
					\noindent\rule{13cm}{0.4pt} \\
					PostDoc at Tata Institute of Fundamental Research (TIFR),  \hfill September 2013 \\
					Mumbai	\hfill	to March 2014
					\noindent\rule{13cm}{0.4pt} \\					
					Visited Dr. Klaus-J\"orn Lange, Wilhelm-Schickard-Institut \hfill March 2012 to\\
					f\"ur Informatik, University of T\"ubingen, Germany \hfill August 2012 
					\noindent\rule{13cm}{0.4pt} \\
					PhD in Theoretical Computer Science at \\
					Institute of Mathematical Sciences(IMSc), Chennai \\
					Advisor: Prof. Kamal Lodaya \hfill August 2007 to \\
					Thesis: Regular Quantifiers in Logics \hfill September 2013 \\
					\noindent\rule{13cm}{0.4pt} 
									
\section{Teaching Experience}
					Took a one semester course titled ``\emph{Verification}" for Graduate \hfill March 2012 to  \\
					and Undergraduate students in \emph{Wilhelm-Schickard-Institut} \hfill August 2012\\ 
					\emph{f\"ur Informatik, University of T\"ubingen, Germany} \\
					\noindent\rule{13cm}{0.4pt} \\
					One week course on ``\emph{Descriptive Complexity theory: An introduction}", \hfill 21 March \\
					 Indian School on Logic and its Applications (ISLA),   \hfill to 26 March, 2016 \\
					 PSG college of Technology, Coimbatore \\
					\noindent\rule{13cm}{0.4pt} 

\section{Industry Experience}
Software Engineer, Infosys Technologies, India \hfill August 2004-July 2005


\section{Awards}
\emph{ACM India Doctoral Dissertation Award 2014: Honourable mention} for doctoral dissertation, ``Regular quantifiers in Logics".

\section{Education} 
					MTech in Computer Science \hfill August 2005 to \\
					Institute: Indian Institute of Technology (IIT), Madras \hfill May 2007 \\
					Project guide: Prof. C. Chandra Sekhar \\
					CGPA: 8.86 \\
					\noindent\rule{13cm}{0.4pt} \\
					BTech in Computer Science \hfill August 2000 to \\
					Institute: College of Engineering, Thiruvananthapuram \hfill June 2004 \\
					Marks: 73\% \\
					\noindent\rule{13cm}{0.4pt} 
					School: Loyola School, Thiruvananthapuram \\
					12th ISC Marks: 88\% \\
					10th ICSE Marks: 91.2\% \\
					\noindent\rule{13cm}{0.4pt} 

 
\section{Publications}
{\bf 7.} Amaldev Manuel, A. V. Sreejith, ``Two variable logic over countable linear orderings", Accepted in Proceedings of Mathematical Foundations of Computer Science (MFCS), 2016 \\
{\bf 6.} Thomas Colcombet, A. V. Sreejith, ``Limited Set quantifiers over Countable Linear Orderings", Proceedings of the Automata, Languages, and Programming - 42nd International Colloquium (ICALP), 2015, 146-158 \\
{\bf 5.} Kamal Lodaya, A. V. Sreejith, ``Counting quantifiers and linear arithmetic on word models" with Kamal Lodaya in 14th Asian Logic Conference (ALC), 2014. \\
{\bf 4.} V. Arvind, S. Raja, A. V. Sreejith, ``On lower bounds for multiplicative circuits and linear circuits in noncommutative domains", 9th International Computer Science Symposium in Russia (CSR), 2014, 65-76\\
{\bf 3.} Andreas Krebs, A. V. Sreejith, ``Non-definability of Languages by Generalized First-order Formulas over (N, +)", Proceedings of the 27th Annual IEEE Symposium on Logic in Computer Science (LICS), 2012, 451-460.\\
{\bf 2.} A. V. Sreejith, ``Expressive Completeness for LTL With Modulo Counting and Group Quantifiers", Method for Modalities (M4M), 2011, ENTCS Vol-278, 201-214.\\
{\bf 1.} Kamal Lodaya, A. V. Sreejith, ``LTL can be more succinct", Proceedings of 8th International Symposium on Automated Technology for Verification and Analysis (ATVA), 2010, LNCS 6252, 21-24. \\

\section{Unpublished manuscripts}
{\bf 4.} Thomas Colcombet, A. V. Sreejith, ``Logics over Countable Linear Orderings". \\
{\bf 3.} Andreas Krebs, A. V. Sreejith, ``Generalized first order formulas over (N,+)". \\
{\bf 2.} Kamal Lodaya, A. V. Sreejith, ``Two variable logics with counting quantifiers". \\
{\bf 1.} Gaurav K., A. V. Sreejith, Devauchelle O., Sinha R., Metivier F., ``Remote sensing to estimate formative discharge of the Himalayan foreland rivers".


\section{Academic Visits}
Dr. Arkadev Chattopadhyay, Tata Institute of Fundamental \hfill March 2016 \\
 Research (TIFR), Mumbai \\
 					\noindent\rule{13cm}{0.4pt} \\
Prof. Klaus-J\"orn Lange, Wilhelm-Schickard-Institut f\"ur Informatik, \hfill October 2015\\
					University of T\"ubingen, Germany \\							
					\noindent\rule{13cm}{0.4pt} \\
Prof. Klaus-J\"orn Lange, Wilhelm-Schickard-Institut f\"ur Informatik, \hfill March 2012 to\\
					University of T\"ubingen, Germany \hfill August 2012 \\
					\noindent\rule{13cm}{0.4pt} \\
Prof. Klaus-J\"orn Lange, Wilhelm-Schickard-Institut f\"ur Informatik, \hfill November 2011\\
					University of T\"ubingen, Germany \\					
					\noindent\rule{13cm}{0.4pt} \\

\section{References}
			Name: 		Prof. Thomas Colcombet \\
			Institute: 	Laboratoire d{'}Informatique Algorithmique:  \\
			Fondements et Applications (LIAFA), Paris, France	\\
			\emph{E-mail:} {thomas.colcombet@liafa.univ-paris-diderot.fr} \\
			Home Page: http://www.liafa.univ-paris-diderot.fr/$\sim$colcombe/ \\
			\noindent\rule{13cm}{0.4pt} \\					
			Name: 		Prof. Kamal Lodaya (Thesis adviser) \\
			Institute: 	The Institute of Mathematical Sciences (IMSc) \\
			CIT Campus, Taramani, Chennai	\\
			\emph{E-mail:} {kamal@imsc.res.in} \\
			Home Page: http://www.imsc.res.in/$\sim$kamal/ \\
			\noindent\rule{13cm}{0.4pt} \\					
			Name: 		Prof. V. Arvind \\
			Institute: 	The Institute of Mathematical Sciences (IMSc) \\
			CIT Campus, Taramani, Chennai	\\
			\emph{E-mail:} {arvind@imsc.res.in} \\
			Home Page: http://www.imsc.res.in/$\sim$arvind/ \\
			\noindent\rule{13cm}{0.4pt} \\					
\end{resume}
\end{document}




