\documentclass[11pt,a4paper,sans]{moderncv} % Font sizes: 10, 11, or 12; paper sizes: a4paper, letterpaper, a5paper, legalpaper, executivepaper or landscape; font families: sans or roman

\moderncvstyle{banking} % CV theme - options include: 'casual' (default), 'classic', 'oldstyle' and 'banking'
\moderncvcolor{grey} % CV color - options include: 'blue' (default), 'orange', 'green', 'red', 'purple', 'grey' and 'black'

\usepackage[scale=0.75]{geometry} % Reduce document margins
%\setlength{\hintscolumnwidth}{3cm} % Uncomment to change the width of the dates column
%\setlength{\makecvtitlenamewidth}{10cm} % For the 'classic' style, uncomment to adjust the width of the space allocated to your name

%\usepackage{hyperref}
\usepackage{graphicx}

\usepackage{subcaption}
\usepackage{tikz}
\usetikzlibrary{arrows,automata,positioning}

%\title{Curriculum Vitae}

\firstname {Sreejith}
\familyname {A V}

\address{Associate Professor, School of Mathematics and Computer Science, }{Indian Institute of Technology Goa, Farmagudi, Goa, 403401}
\homepage{www.iitgoa.ac.in/~sreejithav}
\email{sreejithav@iitgoa.ac.in}
\phone{8939601670}

\begin{document}
\makecvtitle % Print the CV title
% Center the name over the entire width of resume:
%\moveleft.5\hoffset\centerline{\Large \bf A. V. Sreejith}
% Draw a horizontal line the whole width of resume:
%\moveleft\hoffset\vbox{\hrule width\resumewidth height 1pt}\smallskip
% address begins here
% Again, the address lines must be centered over entire width of resume:
%\moveleft.5\hoffset\centerline{Associate Professor, School of Mathematics and Computer Science}
%\moveleft.5\hoffset\centerline{Indian Institute of Technology Goa, GEC campus, Farmagudi, Goa - 403401}
%\moveleft.5\hoffset\centerline{E-mail: sreejithav@iitgoa.ac.in ~~~~ Phone:+918939601670}
%\moveleft.5\hoffset\centerline{ \href{http://www.iitgoa.ac.in/~sreejithav/} {\tt{www.iitgoa.ac.in/$\sim$sreejithav/}}}
%\moveleft.5\hoffset\centerline{Phone: +918939601670}

\newif \ifpersonal
\personalfalse
 
%\begin{resume}
 
%\section{Broad Subject}  Computer Science, Theoretical Computer Science, Formal methods and Verification
\ifpersonal
\section{Personal Information}
Date of birth: March 21, 1983 \hfill Citizenship: Indian
\fi
%\begin{tabular}{ll}
%Date of Birth: & March 21, 1983 \\
%Gender: & Male \\
%Citizenship: & Indian \\ 
%Permanent Address: & TC 14/1328, Kochu Madom, Vazhuthacaud, \\
% & Thiruvananthapuram, Kerala-695014
%\end{tabular}

\section{Research interest}
\begin{enumerate}
 \item \textbf{Theoretical Computer Science:} Mathematical Logic, Automata theory
 \item \textbf{Formal methods:} Formal Verification, Probabilistic automata, Temporal Logics, Formal methods in machine learning
\end{enumerate}

\section{Current position}
\cventry{}{School of Mathematics and Computer Science}{Associate Professor}{August 2021--present}{\textit{IIT Goa}}{Responsibilities: Teaching undergraduate and graduate students, research, and administrative leadership.}
\section{Research positions}
\cventry{April 2024--July 2024}{Professeur Invit\'e}{University of Bordeaux}{France}{\textit{~}}{~}
\cventry{October 2017--July 2021}{Assistant Professor}{School of Mathematics and Computer Science, IIT Goa}{Goa}{\textit{~}}{~}
\cventry{September 2016--September 2017}{Postdoctoral position}{University of Warsaw, Institute of Informatics}{Poland}{\textit{Mentor: Prof. Miko\l{}aj Boja\'nczyk}}{~}
\cventry{October 2015--August 2016}{Postdoctoral position}{Chennai Mathematical Institute}{Chennai}{\textit{~}}{~}
\cventry{April 2014--September 2015}{Postdoctoral position}{Institut de recherche en informatique fondamentale (IRIF)}{Paris, France}{\textit{Mentor: Prof. Thomas Colcombet}}{~}
\cventry{September 2013--March 2014}{Postdoctoral position}{Tata Institute of  Fundamental Research (TIFR)}{Mumbai}{\textit{}}{~}
\cventry{March 2012--August 2012}{Research visit}{University of T\"ubingen}{Germany}{\textit{Mentor: Prof. Klaus-J\"orn Lange}}{~}

\cventry{August 2007--September 2013}{Ph.D.}{Institute of Mathematical Sciences (IMSc)}{Chennai}{\textit{Advisor: Prof. Kamal Lodaya}}{Thesis: Regular Quantifiers in Logics} % arguments 3 to 6 can be left empty
\section{Education}

\cventry{August 2005--May 2007}{M.Tech in Computer Science}{Indian Institute of Technology (IIT)}{Madras}{\textit{CGPA: 8.6}}{~} % arguments 3 to 6 can be left empty

\cventry{August 2000--June 2004}{B.Tech in Computer Science}{College of Engineering}{Thiruvananthapuram}{\textit{73\%}}{~}

\section{Industry }
Software Engineer, Infosys Technologies, India \hfill August 2004 - July 2005

\section{Awards}
$\bullet$ Awarded a position of ``Professeur Invit\'e'' at University of Bordeaux for 2023-2024. \\
$\bullet$ \href{http://awards.acm.org/award_winners/sreejith_9923861.cfm}{{\textbf{ACM India Doctoral Dissertation Award 2014: Honourable mention}}} for doctoral dissertation, \emph{Regular Quantifiers in Logics}. %The award is given by ACM, India.
% the Association for computing machinery (ACM), India. \\
%The award is for the second best doctoral dissertation in computer science in India for the year 2014. 
%{$\bullet$} First prize in programming contest conducted by Cochin University in 2002. \\
%\emph{Regional Mathematical Olympiad (RMO)} in 1998.


				
\section{Research funding}
\begin{itemize}
 \item \textbf{Indo-French grant (CEFIPRA):} \emph{Formal Verification of Adaptive Control Algorithms} \\
Collaborators: Dr. Vincent Penelle (University of Bordeaux), Dr. Meenakshi D’Souza (IIITB)
\item \textbf{SERB MATRICS:} \emph{Probabilistic pushdown automata}
\item \textbf{Ministry of Earth Science (MoES):} \emph{Remote sensing based method for detecting water discharge in the Ganga and Brahmaputra rivers}, project completed \\
Collaborators: Dr. Gaurav Kumar (Earth Scientist, IISER Bhopal)
\end{itemize}


\section{Ph.D guidance}
\cvitem {Prince Mathew (pre-dissertation over, thesis submission by Dec 2024)} {Equivalence and Learning of Weighted One-Counter Automata subclasses}

\section{B.Tech/M.Tech thesis projects}

% \begin{itemize}
% \item Rashika N (MTech, 2024): Verifying properties of deep neural networks
% \item Sumitra (MTech, 2024): Using simulators to train neural networks
% \item Umang Srivastav (MTech, 2024): Neural network inversion
% \item Ankita Chand (Mtech, 2023): Soil moisture estimation
% \item Vaibhav Kumar Rai (MTech, 2021): Epidemic modelling
% \item Shivam Kumar (BTech, 2020): Detection of one function from a set of functions
% %\item Manika Khare and Himali Goel (BTech, 2020): Automating approaches for identification of the morphology of nano particles from microscopic Images
% \end{itemize}

\cvitem {Rashika N (MTech 2024)}{Verifying properties of deep neural networks}
\cvitem {Sumitra (MTech 2024)} {Using simulators to train neural networks}
\cvitem {Umang Strivastav (MTech 2024)} {Neural network inversion}
\cvitem {Ankita Chand (MTech 2023)} {Soil moisture estimation}
\cvitem {Vaibhav Kumar Rai (MTech 2021)} {Epidemic modelling}
\cvitem {Shivam Kumar (BTech 2020)} {Detection of one function from a set of functions}

%\section{Unpublished Manuscripts}
%{\bf 1.} A. V. Sreejith, ``\emph{Unary counting and Exponentiation function}''.\\
%{\bf 1.} A Baskar, A. V. Sreejith, Ramanathan S. Thinniyam, ``\emph{Modulo Quantifiers over Functional Vocabularies Extending Addition}'', arXiv preprint arXiv:1705.00290 \\
%{\bf 3.} Bruno Guillon, Laure Daviaud, Miko\l{}aj Boja\'nczyk, Vincent Penelle, A. V. Sreejith,``\emph{Undecidability of MSO+ultimately periodic}''. \\
%{\bf 4.} Miko\l{}aj Boja\'nczyk, A. V. Sreejith, ``\emph{Universal Trees and decidability}". \\
%{\bf 5.} Thomas Colcombet, A. V. Sreejith, ``\emph{Logics over Countable Linear Orderings}". \\
%{\bf 6.} Andreas Krebs, A. V. Sreejith, ``\emph{Generalized first order formulas over (N,+)}". \\
%{\bf 7.} Kamal Lodaya, A. V. Sreejith, ``\emph{Modulo counting logics}". \\
%{\bf 8.} Gaurav K., A. V. Sreejith, Devauchelle O., Sinha R., Metivier F., ``\emph{Remote sensing to estimate formative discharge of the Himalayan foreland rivers}".

%\section{Industrial Visits}
%Visited Akash Lal in Microsoft Research (MSR) \hfill 2018 \\
%Visited Tata Research and Development Center (TRDDC), Pune: I also gave a talk titled, ``\emph{Counting problems in propositional logic}". \hfill 2018
%
%\section{Academic Visits}
%Prof. Kamal Lodaya, IMSc, Chennai \hfill 2018 \\ 					\noindent\rule{13cm}{0.4pt} \\
%Prof. Bharat Adsul, IIT Bombay \hfill 2018 \\					\noindent\rule{13cm}{0.4pt} \\
%Prof. Vincent Penelle, LABRI-University of Bordeaux \hfill 2018 				\noindent\rule{13cm}{0.4pt} \\
%Prof. Dr. Arkadev Chattopadhaya, Tata Institute of Fundamental \hfill March 2016 \\
% Research (TIFR), Mumbai \\
% 					\noindent\rule{13cm}{0.4pt} \\
%Prof. Klaus-J\"orn Lange,  \hfill October 2015\\
%					Wilhelm-Schickard-Institut f\"ur Informatik, \hfill and March to August 2012\\
%					University of T\"ubingen, Germany \hfill and November 2011\\							
%					\noindent\rule{13cm}{0.4pt} 

%\section{Some Talks}
%{$\bullet$} ``\emph{MSO extended with bounded and periodic predicates}'', invited speaker at Automata, Concurrency and Timed Systems (ACTS) workshop, CMI Feb-17. \\
%{$\bullet$} Invited to give an one week course (five lectures of one hour each), on ``\emph{Descriptive complexity theory: An introduction}" from 21-26 March, 2016 for Indian School on Logic and its Applications (ISLA),  PSG college of Technology, Coimbatore. \\
%{$\bullet$} ``\emph{Two-Variable logic over countable linear orderings}", Mathematical foundations of computer science, Krakow, August 2016 \\
%{$\bullet$}``\emph{Limited set quantifiers over countable linear orderings}" was given in the following places: ICALP, Kyoto, July 2015; Highlights conference, Prague, September 2015; University of T\"ubingen, October 2015; Chennai Theory Day, April 2016. \\
%{$\bullet$} ``\emph{Non-definability of languages by generalized first-order formulas over ($\mathbb{N},+$)}" was given in the following places: University of Stuttgart, Germany, April 2012; LICS, Dubrovnik, Croatia, June 2012; Highlights conference, Paris, September 2014; Dagstuhl, Saarbrucken, Germany, October 2016. \\
%%{$\bullet$} ``\emph{An introduction to theoretical computer science}": A general audience talk in Maison de L'Inde, Paris, February 2015. \\
%{$\bullet$} ``\emph{Power of multiplication}", Indian Statistical Institute, Chennai, December 2013. \\
%{$\bullet$} ``\emph{Expressive completeness for LTL with modulo counting and group operators}", 7th Methods for Modalities workshop, November 2011, Spain.  \\
%{$\bullet$}``\emph{Logics with counting quantifiers}",  University of T\"ubingen, November 2011.  \\
%%{$\bullet$}``\emph{Deciding presburger arithmetic}", Formal Methods Update Meeting, July 2011, VIT University, Vellore, India.  \\
%%{$\bullet$} ``\emph{LTL can be more succinct}", 8th International Symposium on Automated Technology for Verification and Analysis, September 2010, Singapore. \\
%%{$\bullet$} ``\emph{Automata, treewidth and regular expressions, Formal Methods Update Meeting}", July 2010, Dhirubhai Ambani Institute of Information and Communication Technology, Gandhinagar, India.
%%{$\bullet$} ``\emph{Coding tiling problem by Logic}", Institute Seminar Week, IMSc, March 2010.


\section{Teaching}
					\subsection{ Indian Institute of Technology Goa  ($^*$ - shared course)}
					\begin{enumerate}
					\item {Foundations of theoretical computer science}  (graduate course) \hfill Aug 2024/23
					\item {Compilers$^*$} (undergraduate course) \hfill Jan 2024
					\item {Formal methods in machine learning$^*$} (graduate/undergraduate) \hfill Jan 2024
					\item Randomized algorithms (graduate/undergraduate) \hfill Jan 2023/22
					\item {Data structures and algorithms} (undergraduate course)  \hfill  Aug 2022/21/20
					\item {Advanced data structures and algorithms}  (graduate course)  \hfill Aug 2021
					\item {High dimenional data science} (graduate/undergraduate) \hfill Jan 2021
%					{\scriptsize a.  \href{https://www.youtube.com/playlist?list=PLgOvAyZGFRoRjiDfpvnsWZ63D7U0b1Enw}
%					{\tt{https://www.youtube.com/playlist?list=PLgOvAyZGFRoRjiDfpvnsWZ63D7U0b1Enw}}  \\
%					b.  \href{https://www.youtube.com/playlist?list=PLgOvAyZGFRoSKBxqG9lbhtz9cVeLDsZnp}
%							{\tt{https://www.youtube.com/playlist?list=PLgOvAyZGFRoSKBxqG9lbhtz9cVeLDsZnp}}}					
					%\noindent\rule{13cm}{0.4pt}
					\item {Algorithm Design lab} (graduate course) \hfill  Aug 2020
					%\item {Data structures and algorithms} (undergraduate course) \hfill  Jan-May 2020
%					{\scriptsize \href{https://www.iitgoa.ac.in/~sreejithav/20Jan/data-structures/cs113.html}
%						{\tt{https://www.iitgoa.ac.in/$\sim$sreejithav/20Jan/data-structures/cs113.html }}}
					\item {Combinatorial optimization} (graduate/undergraduate course) \hfill  Jan 2020
%					{\scriptsize \href{https://www.iitgoa.ac.in/~sreejithav/20Jan/combinatorial-optimization/cs520.html}
%					{\tt{https://www.iitgoa.ac.in/$\sim$sreejithav/20Jan/combinatorial-optimization/cs520.html }}}
%					{\scriptsize \href{https://www.iitgoa.ac.in/~sreejithav/19July/logic/cs228.html}
%						{\tt{https://www.iitgoa.ac.in/$\sim$sreejithav/19July/logic/cs228.html}}}
%					``\emph{Model checking and software verification}'' \hfill from July 2019 \\
%					Graduate and Undergraduate students in IIT Goa \hfill to December 2019   \\
%					\noindent\rule{13cm}{0.4pt}
					\item {Logic for computer science} (undergraduate course) \hfill Aug 2019/18
					\item {Advanced algorithms} (undergraduate course) \hfill Jan 2019
%					{\scriptsize \href{https://www.iitgoa.ac.in/~sreejithav/19Jan/AdvancedAlgo/cs315.html}
%					{\tt{https://www.iitgoa.ac.in/$\sim$sreejithav/19Jan/AdvancedAlgo/cs315.html}}}
					\item {Topics in Logic and Automata theory} (graduate course) \hfill Jan 2019
%					{\scriptsize \href{https://www.iitgoa.ac.in/~sreejithav/18July/logic/cs228.html}
%						{\tt{https://www.iitgoa.ac.in/$\sim$sreejithav/18July/logic/cs228.html}}}				
					%\item {Software systems lab (shared)} (undergraduate course)    \hfill Aug-Dec 2018
					\item {Computer Programming$^*$} (undergraduate course)  \hfill Summer 2018
					\item {Computer Networks course and lab} (undergraduate course)  \hfill Jan 2018
%					{\scriptsize \href{https://www.iitgoa.ac.in/~sreejithav/18Jan/networking/cs348.html}
%					{\tt{https://www.iitgoa.ac.in/$\sim$sreejithav/18Jan/networking/cs348.html}}} \\
					\end{enumerate}
					
					\subsection{Other places}
					\begin{enumerate}
					\item 12 hour course: {Formal methods in machine learning} (graduate course) \\
					LaBRI, University of Bordeaux \hfill June 2024
					\item {Verification}  (undergraduate and graduate course)  \\
					University of T\"ubingen, Germany \hfill March-Aug 2012
					\item One week course: {Descriptive Complexity theory: An introduction}  \\
					Indian School on Logic and its Applications (ISLA), \\
					PSG college of Technology, Coimbatore \hfill March 2016
					\end{enumerate}
									

\section{Other activities}
\begin{itemize}
 \item \textbf{Current administrative positions at IIT Goa:} Program chair of Computer Science and Engineering, Chair of Senate student advisory committee (SSAC), Head of Information Technology Services in IIT Goa, and member of various other committees at department and institute level.
 \item Program chair of Indian conference on logic and its applications (ICLA) 2023.
 \item Member of the expert committee for Goa startup promotion cell for 2019-2021.
\end{itemize}

%$\bullet$ Heading the Center for Information Technology Services in IIT Goa \\
%$\bullet$ Member of the expert committee for Goa startup promotion cell

% 					\noindent\rule{13cm}{0.4pt} \\
%Member of the Senate Under-graduate Committee (SUGC). \\
% 					\noindent\rule{13cm}{0.4pt} \\
%Member of the Senate Students Advisory Committee (SSAC)\\
% 					\noindent\rule{13cm}{0.4pt} \\
%Member of the website and intranet maintaining committee. 										

			
%\section{References}
%			Name: 		Prof. Kamal Lodaya (Thesis adviser) \\
%			Institute (Retired from): 	The Institute of Mathematical Sciences (IMSc), \\
%			IV Cross Road, CIT Campus, Taramani, Chennai - 600 113	\\
%			\emph{E-mail:} {kamal@imsc.res.in} \\
%			Home Page: \href{http://www.imsc.res.in/~kamal/}{\tt{http://www.imsc.res.in/$\sim$kamal/}} \\
%			\noindent\rule{13cm}{0.4pt} \\		
%			Name: 		Prof.  Miko\l{}aj Boja\'nczyk \\
%			Institute: 	Institute of Informatics, University of Warsaw, MIMUW, \\
%			Banacha 2, 02-097 Warszawa, Poland \\
%			\emph{E-mail:} {bojan@mimuw.edu.pl} \\
%			Home Page: \href{http://www.mimuw.edu.pl/~bojan/}{\tt{http://www.mimuw.edu.pl/$\sim$bojan/}}\\
%			\noindent\rule{13cm}{0.4pt} \\					
%			Name: 		Prof. Bharat Adsul \\
%			Institute: 	Indian Institute of Technology Goa, \\
%			\emph{E-mail:} {adsul@cse.iitb.ac.in} \\
%			Home Page: \href{https://www.cse.iitb.ac.in/~adsul/}{\tt{https://www.cse.iitb.ac.in/$\sim$adsul/}} \\
%			\noindent\rule{13cm}{0.4pt} \\							
%			Name: 		Dr. Andreas Krebs \\
%			Institute: 	Universit\"at T\"ubingen, \\
%			Wilhelm-Schickard-Institut f\"ur Informatik, Sand 13, D-72076, T\"ubingen, Germany \\
%			\emph{E-mail:} {mail@krebs-net.de} \\
%			Home Page: \href{shorturl.at/qsL46}{\tt{shorturl.at/qsL46}} \\
%			\noindent\rule{13cm}{0.4pt}
			
%			 \href{http://www.uni-tuebingen.de/en/faculties/faculty-of-science/departments/computer-science/lehrstuehle/theoretische-informatik/home/mitarbeiter/dr-andreas-krebs.html}{\tt{www.uni-tuebingen.de/en/faculties/faculty-of-science/departments/computer}}\\
%			\href{http://www.uni-tuebingen.de/en/faculties/faculty-of-science/departments/computer-science/lehrstuehle/theoretische-informatik/home/mitarbeiter/dr-andreas-krebs.html}{\tt{-science/lehrstuehle/theoretische-informatik/home/mitarbeiter/dr-andreas-krebs.html}} 



%\newpage
%\section{Unpublished writings}
%\begin{enumerate}
%\item A. V. Sreejith, Lecture notes: Logic in computer science, \\ {\scriptsize \href{https://www.iitgoa.ac.in/~sreejithav/18July/logic/logicNotes.pdf}
%					{\tt{https://www.iitgoa.ac.in/$\sim$sreejithav/18July/logic/logicNotes.pdf}}}
%\item A. V. Sreejith, Notes on parity games, \\ {\scriptsize \href{https://www.iitgoa.ac.in/~sreejithav/papers/parity.pdf}
%					{\tt{https://www.iitgoa.ac.in/$\sim$sreejithav/papers/parity.pdf}}}
%\item B. Adsul, S. Sarkar, A. V. Sreejith, Block Product for finite monoids with generalized associativity,
% {\scriptsize \href{https://www.iitgoa.ac.in/~sreejithav/papers/blockproductCLO.pdf}
%					{\tt{https://www.iitgoa.ac.in/$\sim$sreejithav/papers/blockproductCLO.pdf}}}
%\item S. Biswas, A. V. Sreejith, Notes on pooled testing, \\ {\scriptsize \href{https://www.iitgoa.ac.in/~sreejithav/papers/pooled.pdf}
%					{\tt{https://www.iitgoa.ac.in/$\sim$sreejithav/papers/pooled.pdf}}}
%\item S. Biswas, A. V. Sreejith, Notes on SIR model, \\ {\scriptsize \href{https://www.iitgoa.ac.in/~sreejithav/papers/sir.pdf}
%					{\tt{https://www.iitgoa.ac.in/$\sim$sreejithav/papers/sir.pdf}}}					
%\end{enumerate}

\section{Publications}
\begin{enumerate}
\item Prince Mathew, Vincent Penelle, A. V. Sreejith, \emph{Learning real-time one-counter automata using polynomially many queries}, 2024, \href{https://arxiv.org/abs/2411.08815}{arXiv:2411.08815}.
\item Prince Mathew, Vincent Penelle, Prakash Saivasan, A. V. Sreejith, \emph{Equivalence of deterministic weighted real-time one-counter automata}. In 11th Indian Conference on Logic and its Applications (ICLA), 2025.
\item Prince Mathew, Vincent Penelle, Prakash Saivasan, A. V. Sreejith, \emph{One deterministic counter automata}, Foundations of Software Technology and Theoretical Computer Science (FSTTCS), 39:1-39:23, 2023
\item Bharat Adsul, Saptarshi Sarkar, A. V. Sreejith, \emph{Algebraic Characterizations and Block Product Decompositions for First Order Logic and its Infinitary Quantifier Extensions over Countable Words}, Journal of Computer and System Sciences (JCSS), 136:302-326, 2023
\item Bharat Adsul, Saptarshi Sarkar, A. V. Sreejith, \emph{First-Order logic and its Infinitary Quantifier Extensions over Countable Words}, Fundamentals of Computation Theory (FCT), 39-52, 2021
\item Kumar Gaurav, Fran\c{c}ois M\'etivier, A V Sreejith, Rajiv Sinha, Amit Kumar, and Sampat Kumar Tandon: \href{https://doi.org/10.5194/esurf-9-47-2021}{\emph{Coupling threshold theory and satellite-derived channel width to estimate the formative discharge of Himalayan foreland rivers}}, Earth Surface Dynamics, 9, 47-70, 2021
\item Mico\l{}aj Boja\'nczyk, Laure Daviaud, Bruno Guillon, Vincent Penelle, A. V. Sreejith \emph{Undecidability of a weak version of MSO+U}, Logical Methods in Computer Science (LMCS), volume 16:1, 2020.
\item Bharat Adsul, Saptarshi Sarkar, A. V. Sreejith, \emph{Block products for algebras over countable words and applications to logic}, Logic in Computer Science (LICS), pages 1-13, 2019.
\item Kamal Lodaya, A. V. Sreejith, \emph{Two variable first order logic with counting quantifiers: complexity results}, Developments in Language Theory (DLT), LNCS 10396, pages 260 - 271, 2017.
\item Amaldev Manuel, A. V. Sreejith, \emph{Two variable logic over countable linear orderings}, Mathematical Foundations of Computer Science (MFCS), LIPIcs 58, pages 66:1-66:13, 2016.
\item Thomas Colcombet, A. V. Sreejith, \emph{Limited Set quantifiers over Countable Linear Orderings}, Proceedings of the Automata, Languages, and Programming - 42nd International Colloquium (ICALP), volume 9135 of LNCS, pages 146-158, 2015.
\item V. Arvind, S. Raja, A. V. Sreejith, \emph{On lower bounds for multiplicative circuits and linear circuits in noncommutative domains}, 9th International Computer Science Symposium in Russia (CSR), volume 8476 of LNCS, pages 65-76, 2014.
\item Andreas Krebs, A. V. Sreejith, \emph{Non-definability of Languages by Generalized First-order Formulas over (N, +)}, IEEE Symposium on Logic in Computer Science (LICS),  pages 451-460, 2012.
\item A. V. Sreejith, \emph{Expressive Completeness for LTL With Modulo Counting and Group Quantifiers}, Electronic Notes in Theoretical Computer Science, ENTCS 278, pages 201-214, 2011.
\item Kamal Lodaya, A. V. Sreejith, \emph{LTL can be more succinct}, Proceedings of Automated Technology for Verification and Analysis (ATVA), LNCS 6252, 21-24, 2010.
\end{enumerate}

\section{Others}
\begin{enumerate}
\item Sumitra Dagar, Sudakshina Dutta, Shitala Prasad, Sreejith A V, \emph{Towards Safer Roads: Utilizing Synthetic Data and Neural Networks to Classify Safe Distances in Driving Scenarios} in 9th International Conference on Computer Vision and Image Processing (CVIP 2024).
\item Mohua Banerjee, A. V. Sreejith, \emph{Logic and Its Applications - 10th Indian Conference, ICLA 2023}, Indore, India, March 3-5, 2023, Editors.
\item Kamal Lodaya, A. V. Sreejith, \emph{Counting quantifiers and linear arithmetic on word models} in 14th Asian Logic Conference (ALC), 2014.
\end{enumerate}



\end{document}




