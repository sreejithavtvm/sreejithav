\documentclass[11pt,a4paper,sans]{moderncv} % Font sizes: 10, 11, or 12; paper sizes: a4paper, letterpaper, a5paper, legalpaper, executivepaper or landscape; font families: sans or roman

\moderncvstyle{classic} % CV theme - options include: 'casual' (default), 'classic', 'oldstyle' and 'banking'
\moderncvcolor{grey} % CV color - options include: 'blue' (default), 'orange', 'green', 'red', 'purple', 'grey' and 'black'

\usepackage[scale=0.75]{geometry} % Reduce document margins
%\setlength{\hintscolumnwidth}{3cm} % Uncomment to change the width of the dates column
%\setlength{\makecvtitlenamewidth}{10cm} % For the 'classic' style, uncomment to adjust the width of the space allocated to your name

%\usepackage{hyperref}

\usepackage{varwidth}
\title{Teaching Plan}
\author{Sreejith A V}
\date{}

\firstname {Sreejith}
\familyname {A V}

\begin{document}
\makecvtitle % Print the CV title
\section{Introduction}
{\hskip 2em}I have been teaching theoretical computer science at IIT Goa since 2018, designing and delivering a wide array of core and elective courses for undergraduate and postgraduate students. My teaching contributions have consistently received excellent feedback, earning me the institute's teaching award three times.

{\hskip 2em}In 2024, I served as a Visiting Faculty at the University of Bordeaux for four months, teaching \textit{Formal Methods in Machine Learning} to an international audience comprising PhD students. Additionally, during my PhD in 2013, I taught a course on \textit{Verification} at the University of Tübingen.

{\hskip 2em}Beyond classroom instruction, I have contributed significantly to developing the BTech and MTech curricula for Computer Science and Engineering at IIT Goa. I also played a key role in designing the Mathematics and Computer Science undergraduate program curriculum.

\section{Teaching Philosophy}
My teaching philosophy has been shaped by years of experience, emphasizing active engagement through question / answer sessions, need for catering to diverse backgrounds of students, need for human touch, and need to use computer as a tool for learning.
\begin{itemize}
    \item \textbf{Question / answer sessions:} I integrate tutorials and questions into lectures, actively involving students in applying concepts. Approximately one-third of the class time is dedicated to solving problems. My experience taught me that \emph{students learn more by doing than by listening to lecture}.

    {\hskip 2em}I emphasize working through practical examples. Any complicated definition / algorithm is illustrated with both positive and negative examples.

    \item \textbf{Students with diverse background:} I recognize the diverse backgrounds of my students. For many, English is not their first language. To address this, I explain uncommon terms and their relevance in computing, For example, \emph{semaphores} is explained with railway singaling system.

    {\hskip 2em}I realized a significant cultural differences between urban and rural students. %This diversity requires thinking through what and how we teach.
    While I was giving an example of a Metro station, it dawned on me that a significant number of students have never seen a Metro station. After that, I took extra care on the kind of examples I use.

%    {\hskip 2em}I try to create an environment where every student, regardless of their background, feels they are gaining value from my lectures. Balancing the needs of struggling students with those of advanced learners is a constant focus, though I acknowledge this can be a challenging task.

    \item \textbf{Assessments:} I design exams that reflect the course content, ensuring that students who attend classes and engage with assignments perform well. Exams are not intended to test exceptional intelligence but to assess understanding of the material taught.

    \item \textbf{Computer as a ``teacher'':} A computer is an excellent teacher, as it provides immediate and precise feedback. I encourage students to write programs to test and solidify their understanding.

    \item \textbf{Accessible resources:} I prepare detailed course plans and recommend standard textbooks for undergraduate courses, providing students with clear guidance. I also develop lecture notes, such as those for \emph{Logic for Computer Science}, available on my website (\href{http://www.iitgoa.ac.in/~sreejithav/18July/logic/logicNotes.pdf}{link}). My \emph{Data Structures and Algorithms} course, recorded during COVID-19, is freely available on YouTube (\href{https://www.youtube.com/playlist?list=PLgOvAyZGFRoSKBxqG9lbhtz9cVeLDsZnp}{link}).

    %I usually have a detailed course plan that is shared with the students. Course plans of some of the courses are attached. I also try to give other means of resources. For undergraduate courses, I try to follow a standard text book. This helps the students to go back, read and come prepared for the next lecture. This commitment to follow the text book gets diluted as we move on to masters and then to PhD students. I have developed comprehensive lecture notes for \textit{Logic for Computer Science} which is freely available on my website for students to reference (\href{http://www.iitgoa.ac.in/~sreejithav/18July/logic/logicNotes.pdf}{see link}). During the COVID-19 pandemic, my \textit{Data Structures and Algorithms} course was recorded and is now publicly accessible on YouTube (\href{https://www.youtube.com/playlist?list=PLgOvAyZGFRoSKBxqG9lbhtz9cVeLDsZnp}{see link}).

    \item \textbf{Mentorship:} Additionally, I have guided multiple students in their bachelor projects. These projects often involve reading and discussing research papers or conducting experiments.

\end{itemize}

\section{Selected Course Examples}
Below are three illustrative examples of courses I have taught, showcasing my approach to curriculum design, and teaching. The detailed course plan for these courses are attached.

\subsection*{1. Data Structures and Algorithms \null\hfill \# CS220}
\textbf{Type:} \emph{Core Undergraduate Course} \null\hfill \textbf{Enrollment:} \emph{90+ students} \\
This second-year course (third semester) caters to students from the B.Tech programs in Computer Science and Engineering (CSE) and Mathematics and Computing (M\&C). As a minor course, it also includes select students from Electrical and Mechanical Engineering.

\begin{enumerate}
 \item \textbf{Programming component:} The course builds upon a first-year Python programming course by introducing students to data structures and algorithms through C/C++. Approximately one-quarter of the course time is dedicated to C/C++.
 \item \textbf{Fun in Programming:} To demonatrate the concepts, I wrote a C/C++ program for reading a Bitmap(BMP) image and displaying it in text mode. This example was recorded and shared on YouTube (\href{https://youtu.be/xB0ifokXdWs}{link}), receiving significant appreciation from students.
\item \textbf{Programming to Theory:} A key emphasis in the course was the smooth transition from programming concepts to data structures. The transition was facilitated by introducing abstraction and data types, leading naturally to the concept of abstract data types. This approach helped students distinguish between the user-facing interface of a data structure and its internal implementation.
\end{enumerate}


\textbf{Topics Covered:}
\begin{itemize}
    \item C/C++ programming: basic C concepts, data types, object oriented programming introduction
    \item Core data structures: Linked lists, stacks, queues, trees, and graphs
    \item Fundamental algorithms: Sorting, searching, and their runtime analysis
\end{itemize}

\subsection*{2. Foundations of Theoretical Computer Science \null\hfill \# CS510}
\textbf{Type:} \emph{Core Graduate Course} \null\hfill \textbf{Enrollment:} \emph{10+ students} \\
The course introduces the foundational principles of theoretical computer science to graduate students. It assumes prior knowledge of undegraduate computer science topics like discrete mathematics.

\textbf{Key Topics:}
\begin{enumerate}
    \item \textbf{Proof Strategies:} Building a foundation in formal proof-writing, a skill many students struggle with despite having conceptual understanding.
    \item \textbf{Science of Computation:}
    \begin{itemize}
        \item Review of finite automata, Myhill-Nerode theorem, and the pumping lemma
        \item Introduction to Turing machines, their robustness, and undecidability (e.g., Halting Problem)
        \item Reduction techniques to show undecidability of other problems
    \end{itemize}
    \item \textbf{Practical Computation:}
    \begin{itemize}
        \item Computational complexity: Classes P and NP, and NP-completeness
        \item Approximation / Randomized algorithms: What to do when you do not have fast algorithms?
    \end{itemize}
    \item \textbf{Finite Computation and Digital Circuits:}
    \begin{itemize}
        \item Digital logic: Universal gates, ADDers and MULTipliers
        \item Introduction to perceptrons and multilayer perceptrons as extensions of circuit models
    \end{itemize}
    \item \textbf{Computational Models that Learn:}
    \begin{itemize}
        \item Feedforward neural networks and their ability to learn, contrasting with other models seen.
    \end{itemize}
\end{enumerate}

The course integrates theoretical foundations with advanced topics like approximation and optimization, concluding with an introduction to machine learning concepts.

\newpage
\subsection*{3. Formal Methods in Machine Learning \null\hfill \# CS528}
\textbf{Type:} \emph{Elective Graduate/Undergraduate Course} \null\hfill \textbf{Enrollment:} \emph{30+ students} \\
This interdisciplinary course bridges formal methods and machine learning, focusing on guarantees of robustness, safety, and unbiased decision-making in ML systems. These guarantees are crucial for applications like autonomous vehicles, job applicant screening, and loan approvals, where errors can have serious consequences.

The course explores how formal methods, traditionally used in program and hardware verification, can be adapted for reasoning about learning models.

\textbf{Key Topics:}
\begin{itemize}
    \item Formal verification techniques for ML models
    \item Ensuring roubstness, safety, and unbiasedness in decision-making
    \item Using formal methods to guide the training process
    \item Advanced applications and case studies in industry
\end{itemize}

This course provides students with a unique perspective, blending theory with practical relevance.

\section{Courses Taught and Designed at IIT Goa}
I have taught a diverse range of courses, including:

\subsection*{Core Undergraduate Courses}
\begin{varwidth}[t]{.8\textwidth}
\begin{itemize}
    \item Data Structures and Algorithms (with lab, \emph{Teaching Award recepient})
    \item Logic for Computer Science (\emph{Teaching Award recepient})
    \item Compilers (with lab, \emph{Teaching Award recepient})
\end{itemize}

\end{varwidth}% <---- Don't forget this %
\hspace{3em}% <---- Don't forget this %
\begin{varwidth}[t]{.5\textwidth}
 \begin{itemize}
    \item Automata Theory
    \item Advanced Algorithms
    \item Computer Networks
\end{itemize}
\end{varwidth}

\subsection*{Core Master’s Courses}
\begin{varwidth}[t]{.5\textwidth}
\begin{itemize}
    \item Foundations of Theoretical Computer Science
\end{itemize}

\end{varwidth}% <---- Don't forget this %
\hspace{6.5em}% <---- Don't forget this %
\begin{varwidth}[t]{.5\textwidth}

 \begin{itemize}
    \item Advanced Data Structures
\end{itemize}
\end{varwidth}

\subsection*{Electives (Undergraduate and Postgraduate)}

\begin{varwidth}[t]{.5\textwidth}
\begin{itemize}
    \item Formal Methods in Machine Learning
    \item Combinatorial optimization
\end{itemize}

\end{varwidth}% <---- Don't forget this %
\hspace{10em}% <---- Don't forget this %
\begin{varwidth}[t]{.5\textwidth}
 \begin{itemize}
    \item High-Dimensional Data Science
    \item Advanced Logic (PhD level course)
\end{itemize}
\end{varwidth}% <---- Don't forget this %


\section{Future Teaching Goals}
\begin{itemize}

\item \textbf{New courses:} I aim to expand my teaching portfolio with foundational courses such as Computer Programming, Linear Algebra, Algorithm Design, and Discrete Mathematics. Additionally, as I deepen my understanding of machine learning, I plan to develop a course on its theoretical foundations, including deep learning.

\item \textbf{Course Evolution:} I will expand \emph{Formal Methods in Machine Learning} to incorporate the latest research and innovative applications, keeping the content relevant to industry and academia.


\item \textbf{Improving Learning outcomes:} I am also keen to explore modern pedagogical tools and innovative assessment methods to enhance the student learning experience. \\
\end{itemize}

I look forward to contributing to your institute by teaching and engaging with your students.

\end{document}

