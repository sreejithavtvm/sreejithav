\newpage
\section{Recap Statistics}
%Let $X \sample \ Bernoulli(p)$ where 
%\[
%X = \begin{cases}
%1 \text{ with probability $p$} \\
%0 \text{ with $1-p$}
%\end{cases}
%\]
%
%Let $M_n = \frac{1}{n} \sum_{i=1}^n X_i$. As we take more and more $n$, we tend towards a normal distribution of $M_n$ with mean $p$.
\begin{theorem}[""weak law of large numbers""]
Let $X_1,X_2,\dots,X_n$ be identical random variables such that $\Exp{X_i} = \mu$. Then, for all $\epsilon, \delta > 0$ there exists an $N$ s.t. 
\[
\Prob{ \abs{\frac{1}{n} \sum_{i=1}^n X_i - \mu } > \epsilon}\ <\ \delta \tag{$\forall n \geq N$}
\]
\end{theorem}

The central limit theorem informally says the following: $\frac{1}{n} \sum_{i=1}^n X_i$ approximates the normal distribution $\ndist(p, p(1-p)/n)$.

\begin{lemma}[""Hoeffding bound""]
\label{lem:chernoff}
Let $X_1,X_2,\dots,X_n$ be identical random variables such that $\Prob{a \leq X_i \leq b} = 1$ for all $i \leq n$. Let $\Exp{X_i} = \mu$. Then,
\[
\Prob {\abs{\frac{1}{n} \sum_{i=1}^n X_i - \mu } \geq \epsilon}\ \leq\ 2 e^{-2 n \epsilon^2/(b-a)^2}
\]
\end{lemma}

\AP
\begin{lemma}[""Union bound""]
\label{lem:unionbound}
Let $A$ and $B$ be two events. Then
\[
\Prob {A \cup B}\ \leq\ \Prob A + \Prob B
\]
Let $A_1,A_2,\dots,A_n$ be $n$ events. Then,
\[
\Prob {\bigcup_{i=1}^n A_i}\ \leq\ \sum_{i=1}^n \Prob {A_i}
\]
\end{lemma}

\AP
\begin{lemma}[""Chebyshev inequality""]
Let $X_1,X_2,\dots,X_n$ be iid random variables, where $\Exp{X_i} = \mu$ and $Var(X_i) = \sigma^2$ for all $i \leq n$. Then
\[
\Prob{\abs{\frac{1}{n} \sum_{i=1}^n X_i - \mu} > c}\ \leq\ \frac{\sigma^2}{nc^2}
\]
If in addition $X_i$s are Bernoulli random variables and $\mu \in [0,1]$
\[
\Prob{\abs{\frac{1}{n} \sum_{i=1}^n X_i  - \mu} > c} \ <\ \frac{\mu(1-\mu)}{nc^2}\ \leq\ \frac{1}{4nc^2}
\]
\end{lemma}

\AP We denote the ""indicator random variable"" for an event $E$ by $\intro[\indrand]{}\indrand{E}$. The ""Rademacher random variable"" ($\intro{\Ra}$) is defined as
\[
\sigma = \begin{cases}
-1, \quad \text{ with probability } \frac{1}{2} \\
+1, \quad \text{ with probability } \frac{1}{2}
\end{cases}
\]
