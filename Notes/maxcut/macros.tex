
\newtheorem{theorem}{Theorem}
\newtheorem{corollary}{Corollary}
\newtheorem{definition}{Definition}
\newtheorem{exercise}{Exercise}
\newtheorem{example}{Example}
\newtheorem{puzzle}{Puzzle}
\newtheorem*{solution}{Solution}

\newtheorem{innercustomgeneric}{\customgenericname}
\providecommand{\customgenericname}{}
\newcommand{\newcustomtheorem}[2]{%
  \newenvironment{#1}[1]
  {%
   \renewcommand\customgenericname{#2}%
   \renewcommand\theinnercustomgeneric{##1}%
   \innercustomgeneric
  }
  {\endinnercustomgeneric}
}

\newcustomtheorem{customthm}{Theorem}
\newcustomtheorem{customlemma}{Lemma}
\newtheorem{lemma}[theorem]{Lemma}
\newtheorem{claim}[theorem]{Claim}

\definecolor{light-gray}{gray}{0.95}
\newcommand{\odd}{{\normalfont{Odd}}\xspace}
\newcommand{\even}{{\normalfont{Even}}\xspace}
\newcommand{\finocc}[1]{\mathrm{FIN}(#1)\xspace}
\newcommand{\infocc}[1]{\mathrm{INF}(#1)\xspace}

\newcommand{\Nat}{\mathbb{N}}
\newcommand{\Reals}{\mathbb{R}}
\newcommand{\Rn}{\Reals^n}
\newcommand{\closedset}[2]{[#1, #2]}

\renewcommand{\vec}[1]{\mathbf #1}
\newcommand{\mat}[1]{{\mathbb #1}}
\newcommand{\trans}[1]{{#1}^{\mathsf{T}}}
\newcommand{\rowvec}[1]{\trans{\vec {#1}}}
\newcommand{\dotprod}[2]{\rowvec{#1}  \vec{#2}}
\newcommand{\minimize}{\mathtt{minimize}}
\newcommand{\maximize}{\mathtt{maximize}}

\newcommand{\Imatrix}{\mat{I}}
\newcommand{\cspace}[1]{C(\mat #1)}
\newcommand{\defs}{::=}
\newcommand{\BFS}{\ensuremath{\mathrm{BFS}}\xspace}

\newcommand{\lpstd}[4]{\begin{align*}
\minimize ~\dotprod{#2}{#1} \\
\mat{#3} \vec {#1} \leq \vec #4 \\
\vec #1 \geq \vec 0
\end{align*}}
\newcommand{\lp}[4]{\begin{align*}
\minimize ~\dotprod{#2}{#1} \\
\mat{#3} \vec {#1} = \vec #4 \\
\vec #1 \geq \vec 0
\end{align*}}

\makeatletter
% USAGE \Matrix { a,..,z; A,.., Z ; ... ; aA, ..., zZ}
% NO semi-colon for the last row.
\newcommand{\Matrix}[1]
    {\begin{pmatrix}
      \Matrix@r #1;\@bye;\Matrix@r
     \end{pmatrix}}

\def\Matrix@r #1;{\@bye #1\Matrix@z\@bye\Matrix@s #1,\@bye, }%
\def\Matrix@s #1,{#1\Matrix@t }%
\def\Matrix@t #1,{\@bye #1\Matrix@y\@bye\@firstofone {&#1}\Matrix@t}%
\def\Matrix@y #1\Matrix@t{\\ \Matrix@r }%
\def\Matrix@z #1\Matrix@r {}
\def\@bye  #1\@bye   {}% (the idea of \@bye is from xint code)

\makeatother
\newcommand{\NP}{\ensuremath{\mathrm{NP}}}
\newcommand{\NPX}{\ensuremath{\mathrm{NPX}}}
\newcommand{\Ptime}{\ensuremath{\mathrm{P}}}
\newcommand{\opt}{\ensuremath{\mathrm{OPT}}}
\newcommand{\A}{\ensuremath{\mathcal{A}}}
\newcommand{\Exp}[1]{\ensuremath{\mathtt{E}(#1)}}
\newcommand{\Prob}[1]{\ensuremath{\mathtt{Prob}(#1)}}

%\newcounter{PC} %Problem Counter
%\newcommand{\counter}{\stepcounter{PC}P\arabic{PC}}

%\numberwithin{equation}{section} % This line resets equation numbering when starting a new section.
\renewcommand{\theequation}{P\arabic{equation}} % This line ads "P" in front of your equation numbering.

\newcounter{todocounter}
\newcommand{\notess}[2][]{\stepcounter{todocounter}\todo[color=green!20, #1]{\thetodocounter: #2}}
\newcommand{\sav}[1]{{\color{blue} #1}}
\newcommand{\comment}[1]{{\color{gray} \footnotesize -- \emph{#1} }}

\newcommand{\approxratio}{0.878}

